\documentclass{article}
% 这里是导言区
%\usepackage{indentfirst}%缩进控制
\usepackage{listings}%插入代码
%\usepackage{mcode}
%ctex能够保证能够渲染英文
\usepackage{ctex}
\usepackage{textcomp}
\usepackage{graphicx}%插入图像
\usepackage{epstopdf}
\usepackage{amsmath}
\usepackage{graphicx}
\usepackage{subfigure}
\usepackage{geometry}%设置页边距
\usepackage{amssymb}
\usepackage{float}
\usepackage[level]{datetime} 
\makeatletter
\newcommand{\rmnum}[1]{\romannumeral #1}
\newcommand{\Rmnum}[1]{\expandafter\@slowromancap\romannumeral #1@}
\makeatother
% \renewcommand\thesection{\roman{subsection}}
%\newdateformat{ukdate}{\ordinaldate{\THEDAY} \monthname[\THEMONTH]

\geometry{a4paper,scale=0.75}


\lstset{
tabsize=4, %tab 空格数
frame=shadowbox, %把代码用带有阴影的框圈起来
rulesepcolor=\color{red!20!green!20!blue!20}, %代码块边框为淡青色
keywordstyle=\color{blue!90}\bfseries, %代码关键字的颜色为蓝色, 粗体
showstringspaces=false, %不显示代码字符串中间的空格标记
stringstyle=\ttfamily, %代码字符串的特殊格式
keepspaces=true, %
breakindent=22pt, %
numbers=left, %左侧显示行号
stepnumber=1, %
numberstyle=\tiny, %行号字体用小号
basicstyle=\footnotesize, %
showspaces=false, %
flexiblecolumns=true, %
breaklines=true, %对过长的代码自动换行
breakautoindent=true, %
breakindent=4em, %
aboveskip=1em, %代码块边框
}

\title{Chapter9}
\author{31202008881        \quad \quad \quad
          Bao Ze an}

\begin{document}
\setlength{\parindent}{2em}
\maketitle

\section*{9.1}
for $D(s)=s^2-1$,$N(s)=s-2$,because $D(s)$ and $N(s)$ are coprime.The solution is exist in
\[A(s)D(s)+B(s)N(s)=s^2+2s+2\]
from the example 9.1,we have for any polynomial $Q(s)$
\[
\begin{split}
A(s)=\frac{1}{3}(s^2+2s+2)+Q(s)(-s+2)\\
B(s)=\frac{-1}{3}(s+2)(s^2+2s+2)+Q(s)(s^2-1)
\end{split}
\]
is a solution
we can't find a solution with $\rm{deg} B(s) \leq \rm{deg} A(s)$ in the equation.\\
for any $Q(s)=q_0$ of degree 0,we have $\rm{deg} B(s) >\rm{deg} A(s)$.\\
for any $Q(s)=q_0+q_1s$ of degree 1,we have $\rm{deg} B(s)>\rm{deg} A(s)$.\\for any
$Q(s)=q_0+q_1s+q_2s^2$ of degree 3,we also have $\rm{deg} B(s)>\rm{deg} A(s)$.\\
proceeding forward,we can 
conclude that there exist no solution with $\rm{deg} B(s) \leq \rm{deg} A(s)$ in the equation.

\section*{9.2}
for $F(s)=(s+2)(s+1+j1)(s+1-j1)=s^3+4s^2+6s+4$,$N(s)=s-1$,$D(s)=s^2-4$
$N(s)$ and $D(s)$ are coprime.
\[
\begin{split}
D_0=-4,D_1=0,D_2=1\\
N_0=-1,N_1=1,N_2=0\\
\end{split}    
\]
Suppose 
\[
    \begin{split}
    A(s)=A_0+A_1s \\
    B(s)=B_0+B_1s
    \end{split}
\]
\[
   \left[ 
       \begin{array}{cccc}
        A_0 & B_0 & A_1 & B_1
    \end{array}
   \right]
   \left[ 
       \begin{array}{cccc}
        -4 & 0 & 1 & 0\\
        -1 & 1 & 0 & 0\\
        0 & -4 & 0 & 1\\
        0 & -1 & 1 & 0
    \end{array}
   \right]=\left[ 
       \begin{array}{cccc}
        4 & 6 & 4 & 1
    \end{array}
   \right]
\]
$\Rightarrow \left[ 
    \begin{array}{cccc}
    A_0 & B_0 & A_1 & B_1
    \end{array}
\right]=\left[
    \begin{array}{cccc} 
    -6 & 20 &1 & 10
    \end{array}
\right]
$\\
The compensator equation is $C(s)=\frac{B(s)}{A(s)}=\frac{10s+20}{s-6}$\\
To track any step reference input,$\hat{g}_0(0)=1$,so $p=\frac{F_0}{B_0N_0}=\frac{4}{20 \times -1 }=-0.2$.

\section*{9.3}
for the compensator computed in the problem 9.2,if the transfer function changes to $\overline{\hat{g}}(s)=\frac{s-0.9}{s^-4.1}$
\[
    \begin{split}
    \hat{g}_0(s)=&\frac{pB(s)N(s)}{A(s)D(s)+B(s)N(s)}\\    
          =&\frac{-0.2(20+10s)(s-0.9)}{(s-6)(s^2-4.1)+(10s+20)(s-0.9)}
\end{split}
\]
\[ 
    \hat{g}_0(0)=\frac{3.6}{6.6}=0.55
\]
so the overall system can't track asymptotically any step reference input.
\begin{enumerate}
    \item The compensator of degree 3 \\
    First,we introduce a internal model $\frac{1}{\Phi(s)}=\frac{1}{s}$,Then $B(s)/A(s)$ can be solved form:
    \[A(s)D(s)\Phi(s)+B(s)N(s)=F(s)\]
    Because $\overline{D}(s)=D(s)\Phi(s)$ has degree 3,we may select $A(s)$ and $B(s)$ to have degree 2.Then $F(s)$ has degree 5.\\
    $F(s)=(s+2)(s+1+j1)(s+1-j1)(s+3)(s+3)=s^5+10s^4+39s^3+76s^2+78s+36$
    \[ 
        \begin{split}
            &\overline{D}(s)=D(s)\Phi(s)=(s^2-4.1)s=0-4.1s+0 \dot s^2+1 \dot s^3\\
            &N(s)=s-0.9=-0.9+1 \dot s+0 \dot s^2+0 \dot s^3    
        \end{split}
    \]
    \[ 
    \left[ 
        \begin{array}{cccccc}
            A_0 & B_0 & A_1 & B_1 & A_2 & B_2
        \end{array}
    \right]
    \left[ 
        \begin{array}{cccccc}
            0 & -4.1 & 0 & 1 & 0 & 0\\
            -0.9 & 1 & 0 & 0 & 0& 0\\
            0 & 0 & -4.1 & 0 & 1 & 0\\
            0 & -0.9 & 1 & 0 & 0 & 0\\
            0 & 0 &0 & -4.1 & 0 & 1\\
            0 & 0 & -0.9 & 1 & 0 & 0
        \end{array}
    \right]=\left[ 
        \begin{array}{cccccc}
            36 & 78 & 76 & 39 & 10 &1
        \end{array}
    \right]    
    \]
    $\Rightarrow     \left[ 
        \begin{array}{cccccc}
            A_0 & B_0 & A_1 & B_1 & A_2 & B_2
        \end{array}
    \right]=    \left[ 
        \begin{array}{cccccc}
            -78.5 & -40 & 10 & 226.4 & 1 & 121.6
        \end{array}
    \right]$
    The compensator is :
    \[ 
        C(s)=\frac{B(s)}{A(s)\Phi(s)}=\frac{121.6s^2+226.4s-40}{(s^2+10s-78.5)s}    
    \]
    \item The compensator of degree 2\\
     through using the free parameter, we may able to include an internal model in the compensator
    selecting the $F(s)$ have degree 4.\\
    $F(s)=(s^3+4s^2+6s+4)(s+3)=s^4+7s^3+18s^2+22s+12$
    \[ 
        \begin{split}
            &\overline{D}(s)=D(s)\Phi(s)=(s^2-4.1)=-4.1+0 \dot s+1 \dot s^2\\
            &N(s)=s-0.9=-0.9+1 \dot s+0 \dot s^2    
        \end{split}
    \]
    \[ 
        \left[ 
            \begin{array}{cccccc}
                A_0 & B_0 & A_1 & B_1 & A_2 & B_2
            \end{array}
        \right]
        \left[ 
            \begin{array}{ccccc}
                -4.1 & 0 & 1 & 0 & 0\\
                -0.9 & 1 & 0 & 0 & 0\\
                0 & -4.1 & 0 & 1 & 0\\
                0 & -0.9 & 1 & 0 & 0\\
                0 & 0 & -4.1 & 0 & 1\\
                0 & 0 & -0.9 & 1 & 0
            \end{array}
        \right]=\left[ 
            \begin{array}{ccccc}
                12 & 22 & 18 & 7 &1
            \end{array}
        \right]    
        \] 
    in order for proper compensator,$C(s)=\frac{B_0+B_1s+B_2s^2}{A_0+A_1s+A_2s^2}$,to have $1/s$ as a factor,we require $A_0=0$.
    $\Rightarrow     \left[ 
        \begin{array}{cccccc}
            A_0 & B_0 & A_1 & B_1 & A_2 & B_2
        \end{array}
    \right]=    \left[ 
        \begin{array}{cccccc}
            0 & -13.3 & -18.5 & 45.1 & 1 & 25.5
        \end{array}
    \right]$
    thus the compensator is:
    \[C(s)=\frac{B(s)}{A(s)}=\frac{25.5s^2+45.1s-13.3}{s^2-18.5s}\]
\end{enumerate}

\section*{9.4}
\[ 
\hat{g}(s)=\frac{(s-1)}{s(s-2)}    
\]
There is no need to introduce a feedforward gain to achieve tracking of any step reference input.
Because $D(s)=s(s-2)$,for $r(s)=\frac{a}{s}=\frac{N_r(s)}{D_r(s)}$
the unstable roots of $D_r(s)=s$ is canceled by $D(s)=s(s-2)$,so it has achieved robustly tracking any step reference input.

\section*{9.5}
yes,the design is robust.

\section*{9.6}
for $\hat{g}(s)=\frac{1}{s-1}$,$w(t)=asin(2t+\theta)$\\
in order to achieve the design,the polynomial A(s) must contain the disturbancce model $(s^2+4)$ and the step input $s$.\\
consider $A(s)D(s)+B(s)N(s)=F(s)$,for this equation,we have deg $D(s)=n=1$,thus if $m=n-1=0$, then the solution is unique and we have no freedom in assigning $A(s)$.
if m=2,then we have two free parameters that can be used to assign A(s)\\
Let 
\[     
\begin{split}
&A(s)=\tilde{A_0}s(s^2+4)\\
&B(s)=B_0+B_1s+B_2s^2+B_3s^3
\end{split}
\]
Define $\overline{D}(s)=D(s)(s^2+4)s=(s^2-s)(s^2+4)=0-4s+4s^2-s^3+s^4$\\
for $\tilde{A}_0\overline{D}(s)+B(s)N(s)=F(s)$
\[ 
    \left[ 
        \begin{array}{ccccc}
            \tilde{A}_0 & B_0 & B_1 & B_2 & B_3
        \end{array}
    \right]
    \left[ 
        \begin{array}{ccccc}
            \overline{D}_0 & \overline{D}_1 & \overline{D}_2 & \overline{D}_3 &\overline{D}_4\\
            N_0 & N_1 & 0 & 0 & 0\\
            0 & N_0 & N_1 & 0 & 0\\
            0 & 0 & N_0 & N_1 &0\\
            0 & 0 & 0 & N_0 & N_1
        \end{array}
    \right]=\left[ 
        \begin{array}{ccccc}
            F_0 & F_1 & F_2 & F_3 & F_4
        \end{array}
    \right]    
    \] 
$F(s)=(s+1+j2)(s+1-j2)(s+2+j1)(s+2-j1)=(s^2+2s+5)(s^2+4s+5)=s^4+6s^3+18s^2+30s+25$\\
\[ 
    \left[ 
        \begin{array}{ccccc}
            \tilde{A}_0 & B_0 & B_1 & B_2 & B_3
        \end{array}
    \right]
    \left[ 
        \begin{array}{ccccc}
            0 & -4 & 4 & -1 & 1\\
            1 & 0 & 0 & 0 & 0\\
            0 & 1 & 0 & 0 & 0\\
            0 & 0 & 1 & 0 & 0\\
            0 & 0 & 0 & 1 & 0\\
        \end{array}
    \right]=\left[ 
        \begin{array}{ccccc}
            25 & 30 & 18 & 6 &1
        \end{array}
    \right]    
    \] 
    $\Rightarrow     \left[ 
        \begin{array}{ccccc}
            \tilde{A}_0 & B_0 & B_1 & B_2 & B_3
        \end{array}
    \right]=    \left[ 
        \begin{array}{ccccc}
           1 & 25 & 34 & 14 & 7
        \end{array}
    \right]$
    thus the compensator is:
    \[C(s)=\frac{B(s)}{A(s)}=\frac{7s^3+14s^2+34s+25}{s(s^2+4)}\]

\section*{9.7}
the output can track robustly any step reference input.
Because 
\[ 
\hat{e}(s)=\frac{A(s)D(s)\Phi(s)}{F(s)}\frac{N_r(s)}{D_r(s)}    
\]
because the unstable roots of $D_r(s)=s$ are canceled by $D(s)=s(s-2)$,so we conclude $e(t)\rightarrow 0 \,\rm{as} \, t\rightarrow \infty$
But ,the system can't reject any step disturbancce.
\[
\hat{y}_w(s)=\frac{N(s)A(s)\Phi(s)}{F(s)}\frac{N_w(s)}{D_w(s)}    
\]
the $D_w(s)=s$ can't be canceled.

\section*{9.8}
\[ 
    \hat{g}_{yr}(s)=\frac{1/s}{1+1/s}=\frac{1}{s+1}
    \]
so the transfer function from r to y is BIBO stable,but the system is not totally stable.
the transfer function from $n_1$ to $n_2$ is:
\[
\hat{g}_{n_2n_1}(s)=\frac{1/(s-2)}{1+1/s}=\frac{s}{(s+1)(s-2)}    
\]
is not BIBO stable.

\section*{9.9}
\begin{enumerate}
    \item  the every possible closed-loop transfer function
        \[
            \begin{split}
            &g_{n_2r}=C(s)(1+C(s)\hat{g}(s))^{-1}\\
            &g_{n_3r}=C(s)\hat{g}(s)(1+C(s)\hat{g}(s))^{-1}\\
            & g_{n_3n_2}=\hat{g}(s)(1+C(s)\hat{g}(s))^{-1}\\
            & g_{rn_2}=-\hat{g}(s)(1+C(s)\hat{g}(s))^{-1}\\
            &g_{rn_3}=-(1+C(s)\hat{g}(s))^{-1}\\
           \end{split}
            \]
        so the closed-loop transfer function of every possible input-output pair contains the factor $(1+C(s)\hat{g}(s))^{-1}$.
    \item $(1+C(s)\hat{g}(s))^{-1}$ is proper $\Leftrightarrow$ $|(1+C(\infty)\hat{g}(\infty))^{-1}|<\infty$ $\Leftrightarrow$ $(1+C(\infty)\hat{g}(\infty)) \neq 0$
    \item if $C(\infty)\hat{g}(\infty) \neq -1$,then every possible input-output pair transfer function is a product of two proper fucntion.Then every closed-loop transfer function is proper and well posed.
\end{enumerate}

\section*{9.10}
from the corollary 9.4,we can find\\
$\frac{s-1}{(s+1)^2}$ and $\frac{(s-1)(b_0s+b_1)}{(s+2)^2(s^2+2s+2)}$ is implementable,the others are not implementable.

\section*{9.11}
For $\hat{g}(s)=\frac{(s-1)}{s(s-2)}$ and $\hat{g}_0(s)=\frac{-2(s-1)}{s^2+2s+2}$\\
open loop configurations:\\
\[C(s)=\frac{\hat{g}_0(s)}{\hat{g}(s)}=\frac{\frac{-2(s-1)}{s^2+2s+2}}{\frac{(s-1)}{s(s-2)}}=\frac{-2s(s-2)}{s^2+2s+2}\]
because it involves unstable pole-zero cancellations of $(s-1)$,so it is not totally stable,the implementations can't be used in practice.\\

unity-feedback configurations:
The general form of a unity-feedback configurations is:$\hat{g}_0(s)=\frac{c(s)g(s)}{1+c(s)g(s)}$\\
we can compute the compensator as:
\[C(s)=\frac{\hat{g}_0(s)}{\hat{g}(s)[1-\hat{g}_0(s)]}=\frac{-2(s-2)}{s+4}\]
it also involves unstable pole-zero cancellations of $(s-1)$,so it is not totally stable,it can't be used in practice.

\section*{9.12}
First we compute:
\[\frac{\hat{g}_0(s)}{N(s)}=\frac{-2(s-1)}{(s^2+2s+2)(s-1)}=\frac{-2}{(s^2+2s+2)}=\frac{\overline{E}(s)}{\overline{F}(s)}\]
Because the degree of $\overline{F}(s)$ is 2,we introduce $F(s)=s+3$,so that the degree of $\overline{F}(s)\hat{F}(s)=(s^2+2s+2)(s+3)$ is 3=2n-1\\
thus we have
\[\hat{g}_0(s)=\frac{\overline{E}(s)N(s)\hat{F}(s)}{\overline{F}(s)\hat{F}(s)}=\frac{L(s)N(s)}{A(s)D(s)+M(s)N(s)}\]
\[L(s)=\overline{E}(s)\hat{F}(s)=-2\]
and $A(s)$ and $M(s)$ can be solved from 
\[A(s)D(s)+M(s)N(s)=\overline{F}(s)\hat{F}(s)=(s^2+2s+2)(s+3)=s^3+5s^2+8s+6\]
\[D(s)=s(s-2)=s^2-2s=0-2s+1.s^2\]
\[N(s)=s-1=-1+1.s+0.s^2\]
\[
\left[
    \begin{array}{cccc}
        A_0 & M_0 & A_1 & M_1
    \end{array}
\right]   
\left[ 
        \begin{array}{cccc}
            0 & -2 & 1 & 0\\
            -1 & 1 & 0 & 0\\
            0 & 0 & -2 & 1\\
            0 & -1 & 1 & 0\\
        \end{array}
\right]=\left[ 
    \begin{array}{cccc}
        6 & 8 & 5 & 1
    \end{array}
\right] 
\]
$\Rightarrow 
\left[
    \begin{array}{cccc}
    A_0 & M_0 & A_1 & M_1
    \end{array}
\right]=
\left[
    \begin{array}{cccc}
        -21 & -6 & 1 & 28
    \end{array}
\right]
$\\
thus we have $A(s)=s-21$ and $M(s)=28s-6$
then the compensators are:
\[C_1(s)=\frac{L(s)}{A(s)}=\frac{-2(s+3)}{s-21}\]
\[C_2(s)=\frac{M(s)}{A(s)}=\frac{28s-6}{s-21}\]
we can see the $A(s)$ is not a Hurwitz polynomial
we can't implement the two compensators as show in Fig 9.4(a),because the $A(s)$ is not a
Hurwitz polynomial,the output of $C_1(s)$ will grow without bound and the overall system is not totally stable.

Implementing the two compensators in Fig(9.4)d:
\[
\hat{u}(s)=A^{-1}(s)
\left[
    \begin{array}{cc}
    L(s)& -M(s)
    \end{array}
\right]
\left[
    \begin{array}{c}
        r(s)\\
        y(s)
    \end{array}
\right]=
(
\left[
    \begin{array}{cc}
        -2 & 28
    \end{array}
\right]+
\frac{1}{s-21}\left[
    \begin{array}{cc}
        -48 & 582
    \end{array}
\right]
)(\left[
    \begin{array}{c}
        \hat{r}(s)\\
        \hat{y}(s)
    \end{array}
\right])
\]
Its state-space realization is 
\[
\dot{x}=-21x+\left[
    \begin{array}{cc}
        -48 & 582
    \end{array}
\right]\left[
    \begin{array}{c}
        r\\
        y
    \end{array}
\right]
\]
\[ 
y=x+\left[
    \begin{array}{cc}
        -2 & 28
    \end{array}
\right]
\left[
    \begin{array}{c}
        r\\
        y
    \end{array}
\right]
\]

\section*{9.13}
According to the final value theorem,
$r(t)=at$ $\Rightarrow$ $\hat{r}(s)=\frac{a}{s^2}$
\[
\hat{y}(s)=\hat{g}_0(s)\hat{r}(s)=\hat{g}_0(s)\frac{a}{s^2}=
\frac{k_1}{s}+\frac{k_2}{s^2}+(fraction\; of\; the\; poles\; of\; \hat{g}_0(s))
\]
solve the coefficients,according to the residue method:
\[
\begin{split}
k_2=\hat{g}_0(s).\frac{a}{s^2}.s^2 \Bigg|_{s=0}=\hat{g}_0(0).a\\
k_1=\frac{d}{ds}[\hat{g}_0(s).a] \Bigg|_{s=0}=g^{'}_0(0).a\\  
\end{split}  
\]

if $\hat{g}_0(s)$ is BIBO stable, then every pole lies inside the left half plane will approaches 0 as $t\rightarrow \infty$
thus we have 
\[y_{ss}(t)=\lim_{t \rightarrow \infty}y(t)=\mathcal{L}^{-1}[\frac{k_1}{s}+\frac{k_2}{s^2}]=g_0(0)'.a+\hat{g}_0(0).at\]
Thus if the output is to track asymptotically the ramp reference input we require $\hat{g}_0(0)=1$ and $\hat{g}^{'}_0(0)=0$

\section*{9.14}
necessity:
\[\hat{g}_0(0)=1 \Rightarrow \hat{g}_0(0)=\frac{b_0}{a_0}=1 \Rightarrow b_0=a_0\]
\[
\begin{split}    
&\hat{g}^{'}(0)=0 \\
&\hat{g}^{'}_0(s)=\frac{(b_1+2b_2s+\cdots+mb_ms^{m-1})(a_0+a_1s+\cdots+a_ns^n)-(b_0+b_1s+\cdots+b_ms^m)(a_1+2a_2s+\cdots+na_ns^{n-1})}{(a_0+a_1s+\cdots+a_ns^n)^2}
\end{split}
\]

\[
\begin{split}
&\hat{g}^{'}_0(0)=\frac{b_1a_0-b_0a_1}{a_0^2}=0 \Rightarrow\\
&b_1a_0=b_0a_1
\end{split}
\]
because $a_0=b_0$,$a_1=b_1$\\
sufficiency: if $a_0=b_0$ and $a_1=b_1$,$\hat{g}_0(0)=\frac{b_0}{a_0}=1$
\[\hat{g}^{'}_0(0)=\frac{b_1a_0-b_0a_1}{a_0^2}=0\]

\section*{9.15}
(1) according to the corollary 9.4
all roots of $s^2+2s+a$ have negative real parts,so we need:
\[
\left\{
\begin{split}
s_1+s_2=-2 \leq 0\\
s_1s_2=a >0
\end{split}
\right.    
\]
all zeros of $N(s)=(s+3)(s-2)$ with zero or positive real parts are retained in $(b_1s+b_0)$\\
\[2b_1+b_0=0 \Rightarrow b_0=-2b_1 \]

(2)
\[
\hat{g}_0(0)=\frac{-2b_0}{2 \times 2}=1    
\]
so $b_0=-2$\\
from problem 9.13,we can get:
\[
    \hat{g}^{'}_0(s)=\frac{[b_1s+b_0+(s-2)b_1][(s+2)(s^2+2s+2)]-[(s-2)(b_1s+b_0)][(s^2+2s+2)+(s+2)(2s+2)]}{[(s+2)(s^2+2s+2)]^2}
\]
\[
    \hat{g}^{'}_0(0)=\frac{16b_0-8b_1}{4}=0
\]
$2b_0=b_1$ so $b_1=-4$\\

\section*{9.16}
\[
\hat{G}(s)=\left[
    \begin{array}{c}
        \frac{s+1}{s(s+1)}\\
        \frac{1}{s^2-1}
    \end{array}
\right]=
\left[
    \begin{array}{c}
    (s+1)^2\\
    s
    \end{array}
\right]
\left[
    s(s^2-1)^{-1}
\right]
\]
\[
\begin{split}
&D(s)=s(s^2-1)=s^3-s=0-1.s+0.s^2+1.s^3\\
&N(s)=
\left[
\begin{array}{c}
s^2+2s+1\\
s
\end{array}
\right]=\left[
    \begin{array}{c}
        1\\
        0
    \end{array}
\right]+
\left[
    \begin{array}{c}
        2\\
        1
    \end{array}
\right]s+
\left[
    \begin{array}{c}
        1\\
        0
    \end{array}
\right]s^2+
\left[
    \begin{array}{c}
        0\\
        0
    \end{array}
\right]s^3
\end{split}
\]
\[
S_m=
    \left[ 
        \begin{array}{cccccc}
            0 & -1 & 0 & 1 & 0 & 0\\
            1 & 2 & 1 & 0 & 0& 0\\
            0 & 1 & 0 & 0 & 0 & 0\\
            0 & 0 & -1 & 0 & 1 & 0\\
            0 & 1 &2 & 1 & 0 & 0\\
            0 & 0 & 1 & 0 & 0 & 0\\
            0 & 0 & 0 & -1 & 0 & 1\\
            0 & 0 & 1 & 2& 1 & 0 \\
            0 & 0 & 0 & 1 & 0 & 0
        \end{array}
    \right]    
\]
here we need linearly indepedent rows from top to bottom of $S_m$,there we will apply QR decomposition to the transpose of $S_m$.
we see that there are two linearly indepedent $N_1$-rows and one linearly indepedent $N_2$ row.The degree of $G(s)$ is 3 and we have found 
three linearly indepedent N-rows.Therefore there is no need to search further and we have $v_1=2$ and $v_2=1$.Thus the row index is $v=2$.we select 
$m_1=m_2=m=v-1=1$ thus for any column-reduced F(s) of column degrees $m+\mu=4$,we can find a proper compensator
\[F(s)=(s+2)(s+1+j1)(s+1-j1)(s+3)=s^4+7s^3+18s^2+22s+12\]
\[
\left[
    \begin{array}{cccc}
        A_0 & B_0 & A_1 & B_1
    \end{array}
\right]\tilde{S}_1=
\left[
    \begin{array}{cccc}
        A_0 & B_0 & A_1 & B_1
    \end{array}
\right] 
\left[ 
        \begin{array}{ccccc}
            0 & -1 & 0 & 1 & 0 \\
            1 & 2 & 1 & 0 & 0\\
            0 & 1 & 0 & 0 & 0\\
            0 & 0 & -1 & 0 & 1\\
            0 & 1 &2 & 1 & 0 \\
            0 & 0 & 1 & 0 & 0 \\
        \end{array}
\right]=\left[ 
    \begin{array}{ccccc}
        12 & 22 & 18 & 7 & 1
    \end{array}
\right] 
\]
we knew $\tilde{S}_1$ is full column rank,when seraching the row index,we knew that the last of $\tilde{S}_1$ is a linearly depedent row.
we delete the row, and  assign the second column of $B_1$ as 0
using matlab, we have 
\[
\left[
    \begin{array}{cccc}
        A_0 & B_0 & A_1 & B_1
    \end{array}
\right]=
\left[
    \begin{array}{cccccc}
        3.5 & 12 & -2 &1 &3.5 &0
    \end{array}
\right]    
\]
so we can get $A(s)=s+3.5$ and $B(s)=\left[
    \begin{array}{cc}
        3.5s+12 & -2
    \end{array}
\right]
$\\
\[
    \hat{G}(s)=N(s)F^{-1}(s)B(s)=
    \left[
        \begin{array}{c}
            s^2+2s+1\\
            s
        \end{array}
    \right] 
    (s^4+7s^3+18s^2+22s+12)^{-1}
    \left[
        \begin{array}{cc}
            3.5s+12 & -2
        \end{array}
    \right]
\]
\[
    \hat{G}(0)=
    \left[
        \begin{array}{c}
            1\\
            0
        \end{array}
    \right] 
   \frac{1}{12}
    \left[
        \begin{array}{cc}
            12 & -2
        \end{array}
    \right]
=\left[
    \begin{array}{cc}
        1 & -\frac{1}{6}\\
        0 & 0
    \end{array}
\right]
\]
so the system can't track asymptotically any step reference input.

\section*{9.17}
This problem can be seen as s dual problem as problem 9.16
\[
\begin{split}
    &\hat{G}(s)=[s(s^2-1)]^{-1}
    \left[
        \begin{array}{cc}
            (s+1)^2 & s
        \end{array}
    \right]\\
& \overline{D}(s)=s(s^2-1)=s^3-s=0 -1.s+0.s^2+1.s^3\\
& \overline{N}(s)=  \left[
    \begin{array}{cc}
        (s+1)^2 & s
    \end{array}
\right]=\left[
    \begin{array}{cc}
        1 & 0
    \end{array}
\right]
\left[
    \begin{array}{cc}
        2 & 1
    \end{array}
\right]s+
\left[
    \begin{array}{cc}
        1 & 0
    \end{array}
\right]s^2+
\left[
    \begin{array}{cc}
        0 & 0
    \end{array}
\right]s^3
\end{split}    
\]
\[
\tilde{S}_1.\left[
    \begin{array}{c}
        \overline{B}_0\\
        \overline{A}_0\\
        \overline{B}_1\\
        \overline{A}_1
    \end{array}
\right]=\left[
    \begin{array}{cccccc}
            0 & 1 & 0 & 0 & 0 & 0\\
            -1 & 2 & 1 & 0 & 1 & 0\\
            0 & 1 & 0 & -1 & 2 & 1\\
            1 & 0 & 0 & 0 & 1 & 0\\
            0 & 0 & 0 & 1 & 0 & 0\\
    \end{array}
\right]
\left[
    \begin{array}{c}
        \overline{B}_0\\
        \overline{A}_0\\
        \overline{B}_1\\
        \overline{A}_1
    \end{array}
\right]=
\left[
    \begin{array}{c}
        12\\
        22\\
        18\\
        7\\
        1
    \end{array}
\right]
\]
as the procedure in problem 9.16,we can get
\[
C(s)=\overline{B}(s)\overline{A}^{-1}(s)=
\left[
    \begin{array}{c}
        3.5s+12\\
        2
    \end{array}
\right](s+3.5)^{-1}
\]
\[\hat{G}_0(s)=I-\overline{A}(s)\overline{F}^{-1}(s)\overline{D}(s)=1-(s+3.5)(s^4+7s^3+18s^2+22s+12)^{-1}(s^3-s)\]
Because $\hat{G}_0(0)=1$,so we can find a feedforward gain so that the overall system will track asymptotically any step reference input.

\section*{9.18}
\[\hat{G}(s)=
\left[
\begin{array}{cc}
    \frac{s-2}{s^2-1} & \frac{1}{s-1}\\
    \frac{1}{s} & \frac{1}{s-1}
\end{array}
\right]=
\left[
    \begin{array}{cc}
        s(s-2) & 1\\
        s^2-1 & 2
    \end{array}
\right]\left[
    \begin{array}{cc}
        s(s^2-1) & 0\\
        0 & s-1
    \end{array}
\right]^{-1}
=N(s)D^{-1}(s)
\]
It is easily to find N(s) and D(s) are right coprime.
\[
\begin{split}
D(s)& =\left[
    \begin{array}{cc}
        s^3-s & 0\\
        0 & s-1
    \end{array}
\right]=
\left[
    \begin{array}{cc}
        0 & 0\\
        0 & -1
    \end{array}
\right]+
\left[
    \begin{array}{cc}
        -1 & 0\\
        0 & 1
    \end{array}
\right]s+
\left[
    \begin{array}{cc}
        0 & 0\\
        0 & 0
    \end{array}
\right]s^2+
\left[
    \begin{array}{cc}
        1 & 0\\
        0 & 0
    \end{array}
\right]
\\
& =D_0+D_1s+D_2s^2+D_3s^3
\end{split}
\]

\[
\begin{split}
N(s)& =\left[
    \begin{array}{cc}
        s^2-2s & 1\\
        s^2-1 & 2
    \end{array}
\right]=
\left[
    \begin{array}{cc}
        0 & 1\\
        -1 & 2
    \end{array}
\right]+
\left[
    \begin{array}{cc}
        -2 & 0\\
        0 & 0
    \end{array}
\right]s+
\left[
    \begin{array}{cc}
        1 & 0\\
        0 & 0
    \end{array}
\right]s^2+
\left[
    \begin{array}{cc}
        0 & 0\\
        0 & 0
    \end{array}
\right]
\\
& =N_0+N_1s+N_2s^2+N_3s^3
\end{split}
\]
\[
S_m=
    \left[ 
        \begin{array}{cccccc}
            D_0 & D_1 & D_2 & D_3 & 0 & 0\\
            N_0 & N_1 & N_2 & N_3 & 0& 0\\
            0 & D_0 & D_1 & D_2 & D_3 & 0\\
            0 & N_0 & N_1 & N_2 & N_3 & 0\\
            0 & 0 & D_0 & D_1 & D_2 & D_3\\
            0 & 0 & N_0 & N_1 & N_2 & N_3 \\
        \end{array}
    \right]     
\]
we apply QR decomposition to the transpose of $S_m$,we see that there are two linearly indepedent $N_1$-rows and two linearly indepedent $N_2$-rows
The degree of $G(s)$ is 4,there is no need to search further and we have $v_1=2$ and $v_2=2$.we select $m_1=m_2=m=v-1=1$.\\
so for $F(s)$ of column degrees $m+u_1=4$ and $m+u_2=2$.there exists a compensator
choose
\[
\begin{split}F(s)&=
\left[
    \begin{array}{cc}
        s^4+7s^3+18s^2+22s+12 & 0\\
        0 & s^2+2s+2
    \end{array}
\right]=\left[
    \begin{array}{cc}
        12 & 0\\
        0 & 2\\
    \end{array}
\right]+\left[
    \begin{array}{cc}
        22 & 0\\
        0 & 2
    \end{array}
\right]s+\left[
    \begin{array}{cc}
        18 & 0\\
        0 & 1
    \end{array}
\right]s^2 \\
& +\left[
    \begin{array}{cc}
        7 & 0\\
        0 & 0
    \end{array}
\right]s^3+\left[
    \begin{array}{cc}
        1 & 0\\
        0 & 0
    \end{array}
\right]s^4
\end{split}
\]
\[
    \left[ 
        \begin{array}{cccc}
            A_0 & B_0 & A_1 & B_1
        \end{array}
    \right]
    \left[ 
        \begin{array}{ccccc}
            D_0 & D_1 & D_2 & D_3 &0\\
            N_0 & N_1 & N_2 & N_3 & 0\\
            0 & D_0 & D_1 & D_2 & D_3\\
            0 & N_0 & N_1 & N_2 & N_3\\
        \end{array}
    \right]=\left[ 
        \begin{array}{ccccc}
            F_0 & F_1 & F_2 & F_3 & F_4
        \end{array}
    \right]    
\]
using matlab,we can get 
\[
    \left[ 
        \begin{array}{cccc}
            A_0 & B_0 & A_1 & B_1
        \end{array}
    \right]=\left[
        \begin{array}{cccccccc}
            -46.7 & -53.7 & -29.7 & -12 & 1 & 0 & -30.3 & 42\\
            -2.5 & -2 & 0 & 0 & 0 & 1 & 0 & 2.5
        \end{array}
    \right]
\]
Because $B_0$ is singular,so we need to choose a different $F(s)$.\\
we select 
\[
    F(s)=
    \left[
        \begin{array}{cc}
            s^4+7s^3+18s^2+22s+12 & 0\\
            1 & s^2+2s+2
        \end{array}
    \right]    
\]
then we can get 
\[
    \left[ 
        \begin{array}{cccc}
            A_0 & B_0 & A_1 & B_1
        \end{array}
    \right]=\left[
        \begin{array}{cccccccc}
            -4.7 & -53.7 & -29.7 & -12 & 1 & 0 & -30.3 & 42\\
            -3.3 & -4.3 & -0.30 & -1 & 0 & 1 & -0.7 & 4
        \end{array}
    \right]
\]
thus the compensator:
\[
A(s)=\left[ 
    \begin{array}{cc}
        s-4.7 & -53.7\\
        -3.3 & s-4.3
    \end{array}
\right]    
\]
\[ 
    B(s)=\left[ 
        \begin{array}{cc}
            -30.3s-29.7 & 4.2s-12\\
            -0.7s-0.3 & 4s-1
        \end{array}
    \right]   
\]
The feedforward gain to achieve tracking is:
\[ 
P=B^{-1}(0)F(0)N^{-1}(0)=
\left[ 
    \begin{array}{cc}
        -29.7 & -12\\
        -0.3 & -1
    \end{array}
\right]^{-1}
\left[ 
    \begin{array}{cc}
        12 & 0\\
        1 & 2
    \end{array}
\right]
\left[ 
    \begin{array}{cc}
        0 & 1\\
        -1 & 2
    \end{array}
\right]^{-1}
=\left[ 
\begin{array}{cc}
    0.92 & 0\\
    -4.28 & 1
\end{array}
\right]
\]

\section*{9.19}
\[\hat{G}(s)=
\left[
\begin{array}{cc}
    \frac{s-2}{s^2-1} & \frac{1}{s-1}\\
    \frac{1}{s} & \frac{1}{s-1}
\end{array}
\right]
\]
\[\hat{G}_0(s)=
\left[
\begin{array}{cc}
    \frac{4(s^2-4s+1)}{(s^2+2s+2)(s+2)} & 0\\
    0 & \frac{4(s^2-4s+1)}{(s^2+2s+2)(s+2)} 
\end{array}
\right]
\]
\[\hat{T}(s)=\hat{G}^{-1}(s)\hat{G}_0(s)\]
\[
    \begin{split}    
\hat{G}^{-1}(s)&=
\left[
\begin{array}{cc}
  s(s^2-1) & 0 \\
    0 & s-1
\end{array}
\right]\left[ 
    \begin{array}{cc}
    s(s-2) & 1\\
    s^2-1 & 2
    \end{array}
\right]^{-1}=\left[ 
    \begin{array}{cc}
        s(s^2-1) & 0\\
        0 & s-1
    \end{array}
\right]
\left[ 
    \begin{array}{cc}
        \frac{2}{s^2-4s+1} & \frac{-1}{s^2-4s+1}\\
        \frac{1-s^2}{s^2-4s+1} & \frac{s(s-2)}{s^2-4s+1}
    \end{array}
\right] \\
&=\left[ 
    \begin{array}{cc}
        \frac{2s(s^2-1)}{s^2-4s+1} & \frac{-s(s^2-1)}{s^2-4s+1}\\
        \frac{(1-s^2)(s-1)}{s^2-4s+1} & \frac{s(s-1)(s-2)}{s^-4s+1}
    \end{array}
\right]
\end{split}
\]
\[ 
\begin{split}
\hat{T}(s)
& =\left[ 
    \begin{array}{cc}
        \frac{2s(s^2-1)}{s^2-4s+1} & \frac{-s(s^2-1)}{s^2-4s+1}\\
        \frac{(1-s^2)(s-1)}{s^2-4s+1} & \frac{s(s-1)(s-2)}{s^2-4s+1}
    \end{array}
\right]
\left[ 
    \begin{array}{cc}
        \frac{4(s^2-4s+1)}{(s^2+2s+2)(s+2)} & 0\\
        0 & \frac{4(s^2-4s+1)}{(s^2+2s+2)(s+2)}
    \end{array}
\right] \\
&=\left[ 
    \begin{array}{cc}
        \frac{2s(s^2-1)}{(s^2+2s+2)(s+2)} & \frac{-4s(s^2-1)}{(s^2+2s+2)(s+2)}\\
        \frac{4(1-s^2)(s-1)}{(s^2+2s+2)(s+2)} & \frac{4s(s-1)(s-2)}{(s^2+2s+2)(s+2)}
    \end{array}
\right]
\end{split}
\]
\[ 
T(\infty)=\left[ 
    \begin{array}{cc}
        2 & -4\\
        -4 & 4
    \end{array}
\right] 
\]
is proper and BIBO stable,so it is implementbale. 
\[ 
\begin{split}
N^{-1}(s)\hat{G}_0(s)
& =\left[ 
    \begin{array}{cc}
        \frac{2}{s^2-4s+1} & \frac{-1}{s^2-4s+1}\\
        \frac{1-s^2}{s^2-4s+1} & \frac{s(s-2)}{s^2-4s+1}
    \end{array}
\right]
\left[ 
    \begin{array}{cc}
        \frac{4(s^2-4s+1)}{(s^2+2s+2)(s+2)} & 0\\
        0 & \frac{4(s^2-4s+1)}{(s^2+2s+2)(s+2)}
    \end{array}
\right]\\
& =\left[ 
    \begin{array}{cc}
        \frac{8}{(s^2+2s+2)(s+2)} & \frac{-4}{(s^2+2s+2)(s+2)}\\
        \frac{4(1-s^2)}{(s^2+2s+2)(s+2)} & \frac{4s(s-2)}{(s^2+2s+2)(s+2)}
    \end{array}
\right]\\
& =\left[ 
    \begin{array}{cc}
        (s^2+2s+2)(s+2) & 0\\
        0 & (s^2+2s+2)(s+2)
    \end{array}
\right]^{-1}\left[ 
    \begin{array}{cc}
        8 & -4 \\
        4(1-s^2) & 4s(s-2)
    \end{array}
\right]=\overline{F}^{-1}(s)\overline{E}(s)
\end{split}
\]
the degree of $N^{-1}(s)\hat{G}_0(s)$ can easily be computed as 6.The determinant of $\overline{F}(s)$ is 6.Thus the pair $\overline{F}(s)$ and $\overline{E}(s)$ are left coprime.
The row index of $\hat{G}(s)$ was computed as $v=2$\\
Let us select $\hat{F}(s)=\rm{diag}(s+3,1)$ \\
Then we have:
\[
 \hat{F}(s)\overline{F}(s)=
 \left[ 
     \begin{array}{cc}
     (s^2+2s+2)(s+2)(s+3) & 0\\
     0 & (s^2+2s+2)(s+2)
     \end{array}
 \right]   
\]
it is row-column reduced with column degrees $\{\mu_1=3,\mu_2=1\} \}$ and with row degrees $\{m_1=1,m_2=2\}$
\[ 
    \begin{split}
    L(s)=\hat{F}(s)\overline{E}(s)&=
    \left[ 
        \begin{array}{cc}
            s+3  & 0\\
            0 & 1
        \end{array}
    \right]\left[ 
        \begin{array}{cc}
        8 & -4\\
        4(1-s^2) & 4s(s-2)
        \end{array}
    \right]\\
   & =\left[ 
        \begin{array}{cc}
            8(s+3) & -4(s+3)\\
            4(1-s^2) & 4s(s-2)
        \end{array}
    \right]
    \end{split}
\]
and solve $A(s)$ and $M(s)$ from:
\[A(s)D(s)+M(s)N(s)=\hat{F}(s)\overline{F}(s)\]
Note that the next two N rows are known to be linearly depedent,therefore ,they are deleted and consequently $M_2$ must be 
assigned to 0,so the solution is 
\[
\left[
    \begin{array}{cccccccccccc}
        -\frac{14}{3} & -\frac{161}{3} & -\frac{89}{3} & -12 & 1 & 0 & -\frac{-91}{3} & 42 & 0 & 0 & 0 & 0  \\
        -7.5 & -4 & 0 & 0& 0 & 5 & 0 & 7.5 & 0  & 1 & 0 & 0
    \end{array}
\right]    
\]
thus we have
\[
A(s)=\left[ 
    \begin{array}{cc}
        s-\frac{14}{3} & -\frac{161}{3}\\
        -7.5 & s^2+5s-4
    \end{array}
\right]    
\]
\[ 
    M(s)=\left[ 
        \begin{array}{cc}
            -\frac{91}{3}s-\frac{89}{3} & 42s-12\\
            0 & 7.5s
        \end{array}
    \right]   
\]

\end{document}