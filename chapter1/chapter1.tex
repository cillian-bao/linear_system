\documentclass{article}
% 这里是导言区
\usepackage{listings}%插入代码
\usepackage{mcode}
\usepackage{textcomp}
\usepackage{graphicx}%插入图像
\usepackage{epstopdf}
\usepackage{amsmath}
\usepackage{subfigure}
\usepackage{geometry}%设置页边距


\geometry{a4paper,scale=0.75}


\lstset{
tabsize=4, %tab 空格数
frame=shadowbox, %把代码用带有阴影的框圈起来
rulesepcolor=\color{red!20!green!20!blue!20}, %代码块边框为淡青色
keywordstyle=\color{blue!90}\bfseries, %代码关键字的颜色为蓝色, 粗体
showstringspaces=false, %不显示代码字符串中间的空格标记
stringstyle=\ttfamily, %代码字符串的特殊格式
keepspaces=true, %
breakindent=22pt, %
numbers=left, %左侧显示行号
stepnumber=1, %
numberstyle=\tiny, %行号字体用小号
basicstyle=\footnotesize, %
showspaces=false, %
flexiblecolumns=true, %
breaklines=true, %对过长的代码自动换行
breakautoindent=true, %
breakindent=4em, %
aboveskip=1em, %代码块边框
}

\title{Chapter2}
\author{31202008881        \quad \quad \quad
          Bao Ze an}
\date{today}

\begin{document}
\maketitle
\section*{2.1}
What we know is the linear system must obey the superposition property.\\
The input-output description in Fig2.1(a) is :$y=a*u$.\\
Here $a$ is a constant .It is eay to find the system(a) is a linear system.\\
The input-output decription in Fig2.1(b) is:$y=a*u+b$.\\
Here $a$ and $b$ are all constants.Thstify whether the system has the property of additivity
Let:
\[y_1=a*u_1+b.\]
\[y_2=a*u_2+b.\]
then:
\[(y_1+y_2)\]