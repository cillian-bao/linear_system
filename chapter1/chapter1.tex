\documentclass{article}
% 这里是导言区
\usepackage{listings}%插入代码
\usepackage{mcode}
\usepackage{textcomp}
\usepackage{graphicx}%插入图像
\usepackage{epstopdf}
\usepackage{amsmath}
\usepackage{subfigure}
\usepackage{geometry}%设置页边距


\geometry{a4paper,scale=0.75}


\lstset{
tabsize=4, %tab 空格数
frame=shadowbox, %把代码用带有阴影的框圈起来
rulesepcolor=\color{red!20!green!20!blue!20}, %代码块边框为淡青色
keywordstyle=\color{blue!90}\bfseries, %代码关键字的颜色为蓝色, 粗体
showstringspaces=false, %不显示代码字符串中间的空格标记
stringstyle=\ttfamily, %代码字符串的特殊格式
keepspaces=true, %
breakindent=22pt, %
numbers=left, %左侧显示行号
stepnumber=1, %
numberstyle=\tiny, %行号字体用小号
basicstyle=\footnotesize, %
showspaces=false, %
flexiblecolumns=true, %
breaklines=true, %对过长的代码自动换行
breakautoindent=true, %
breakindent=4em, %
aboveskip=1em, %代码块边框
}

\title{Chapter2}
\author{31202008881        \quad \quad \quad
          Bao Ze an}
\date{today}

\begin{document}
\maketitle
\section*{2.1}
What we know is the linear system must obey the superposition property.\\
The input-output description in Fig2.1(a) is :$y=a*u$.\\
Here $a$ is a constant .It is eay to find the system(a) is a linear system.\\
The input-output decription in Fig2.1(b) is:$y=a*u+b$.\\
Here $a$ and $b$ are all constants.Thstify whether the system has the property of additivity
Let:
\[y_1=a*u_1+b.\]
\[y_2=a*u_2+b.\]
then:
\[(y_1+y_2)=a*(u_1+u_2)+2*b\]
so it does not satisfy the property of additivity.therefore,it is a nonlinear system.\\
It is obviously the system in the Fig2.1(c) is a nonliear system.\\
When system(b) introduce $y-y_0$ as the new output,system(c) can be the linear system.\\

section*{2.2}
Because $g(t)$ is not zero,when $t \leq 0$,so the ideal lowpass filter is not causual and the ideal filter 
can't build in the real world.\\

\section*{2.3}
It is easy to find the system is a linear system.\\
Testify whether the system is time-invariable:\\
Definine the initial time of input $t_0$,system input is $u(t)$,$t \geq t_0$,so it decides the output $y(t)$,$t \geq t_0$ \\
%\begin{aligned}
\[y(t)=\left\{
\begin{aligned}
&u(t)  , &for \quad t_0 \leq t \leq \alpha \\
&0  , &for \quad t \geq \alpha.
\end{aligned}
\right.\]
%\end{aligned}
Shift the initial time to $t_0+T$.Let $t_0+T> \alpha$,and shift the input to $u(t-T)$,$t \geq t_0+T$.\\
The system output is $y'(t)=0$.Suppose that $u(t)$ is not 0,$y'(t)$ is not equal to $y(t-T)$.\\
So,the system is time-invaring.\\
For any time t,the system output y(t) is decided by the current input u(t) exclusively.\\
So,it is a causual time.\\

\section*{2.4}
Let:$y=Hu$
Because of causual property:
\[y_{(- \infty,\alpha)}=Hu_{(-\infty,+\infty)}=Hu_{(-\infty,\alpha)}=HP_\alpha u_{(-\infty,+\infty)}\]\\
Thus we have:
\[P_\alpha y=P_\alpha Hu=P_\alpha HP_\alpha u\]
Because$(P_\alpha Hu)(t)=0$for $t\geq\alpha$,but $(HP_\alpha u)(t)$can be nonzero for $t \geq\alpha$
Thus $P_\alpha Hu \neq HP_\alpha u$
\end{document}