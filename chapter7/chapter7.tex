\documentclass{article}
% 这里是导言区
%\usepackage{indentfirst}%缩进控制
\usepackage{listings}%插入代码
\usepackage{mcode}
%ctex能够保证能够渲染英文
\usepackage{ctex}
\usepackage{textcomp}
\usepackage{graphicx}%插入图像
\usepackage{epstopdf}
\usepackage{amsmath}
\usepackage{graphicx}
\usepackage{subfigure}
\usepackage{geometry}%设置页边距
\usepackage{amssymb}
\usepackage{float}
\usepackage[level]{datetime} 
\makeatletter
\newcommand{\rmnum}[1]{\romannumeral #1}
\newcommand{\Rmnum}[1]{\expandafter\@slowromancap\romannumeral #1@}
\makeatother
% \renewcommand\thesection{\roman{subsection}}
%\newdateformat{ukdate}{\ordinaldate{\THEDAY} \monthname[\THEMONTH]

\geometry{a4paper,scale=0.75}


\lstset{
tabsize=4, %tab 空格数
frame=shadowbox, %把代码用带有阴影的框圈起来
rulesepcolor=\color{red!20!green!20!blue!20}, %代码块边框为淡青色
keywordstyle=\color{blue!90}\bfseries, %代码关键字的颜色为蓝色, 粗体
showstringspaces=false, %不显示代码字符串中间的空格标记
stringstyle=\ttfamily, %代码字符串的特殊格式
keepspaces=true, %
breakindent=22pt, %
numbers=left, %左侧显示行号
stepnumber=1, %
numberstyle=\tiny, %行号字体用小号
basicstyle=\footnotesize, %
showspaces=false, %
flexiblecolumns=true, %
breaklines=true, %对过长的代码自动换行
breakautoindent=true, %
breakindent=4em, %
aboveskip=1em, %代码块边框
}

\title{Chapter7}
\author{31202008881        \quad \quad \quad
          Bao Ze an}

\begin{document}
\setlength{\parindent}{0em}
\maketitle

\section*{7.1}
\[\hat{g(s)}=\frac{s-1}{(s^2-1)(s+2)}=\frac{s-1}{s^3+2s^2-s-2}\]
so,from the equation (7.9),we can get the three-dimensional controllable realization:\\
\[
\dot{x}=
\left[
\begin{array}{ccc}
-2 & 1 & 2\\
1 & 0 & 0\\
0 & 1 & 0\\
\end{array}    
\right]x+
\left[
    \begin{array}{c}
    1\\
    0\\
    0
    \end{array}
\right]u
\]
\[
    y=\left[
        \begin{array}{ccc}
            0 & 1 & -1
        \end{array}
        \right]x    
\]
it is obviously the $\hat{g(s)}$ is not coprime fraction,so the controllable realization is not observable.

\section*{7.2}
from the quation (7.14), we can easily get the three-dimensional observable realization:\\
\[
\dot{x}=
\left[
\begin{array}{ccc}
-2 & 1 & 0\\
1 & 0 & 1\\
2 & 0 & 0\\
\end{array}    
\right]x+
\left[
    \begin{array}{c}
    0\\
    1\\
    -1
    \end{array}
\right]u
\]
\[
    y=\left[
        \begin{array}{ccc}
            1 & 0 & 0
        \end{array}
        \right]x    
\]
similarly, it is not controllable.

\section*{7.3}
from the inverse canonical decomposition,we can add an uncontrollable state to problem 7.1:
\[
\dot{x}=
\left[
\begin{array}{cccc}
-2 & 1 & 2 & a_1\\
1 & 0 & 0 & a_2\\
0 & 1 & 0 & a_3\\
0 & 0 & 0 &a_4\\
\end{array}    
\right]x+
\left[
    \begin{array}{c}
    1\\
    0\\
    0\\
    0
    \end{array}
\right]u
\]
\[
    y=\left[
        \begin{array}{cccc}
            0 & 1 & -1 & c_4
        \end{array}
        \right]x    
\]
where $a_i$ and $c_i$ are arbitrary value,this is an uncontrollable and unobservable realization.
the transfer function can be reduced to a coprime fraction,which is:
\[
\hat{g(s)}=\frac{1}{(s+1)(s+2)}=\frac{1}{s^2+3s+2}    
\]
A minimal realization can be realized through controllable realization, this realization is two-dimensional
\[
\dot{x}=
\left[
\begin{array}{cc}
-3 & -2\\
1 & 0\\
\end{array}    
\right]x+
\left[
    \begin{array}{c}
    1\\
    0
    \end{array}
\right]u
\]
\[
    y=\left[
        \begin{array}{cc}
            0 & 1
        \end{array}
        \right]x    
\]

\section*{7.4}
from the equation (7.27),we can get:
\[
\begin{aligned}
\begin{split}
&D(s)=-2-s+2s^2+s^3\\
&N(s)=-1+s\\
&\overline{D}(s)=\overline{D}_0+\overline{D}_1s+\overline{D}_2s^2\\
&\overline{N}(s)=\overline{N}_0+\overline{N}_1s+\overline{N}_2s^2\\
\end{split}
\end{aligned}
\]
the Sylvester resultant:
\[
\textbf{SM}:=
\left[
\begin{array}{cccccc}
-2 &-1 & 0  & 0 & 0 & 0\\
-1 &1  & -2 & -1& 0 & 0\\
2  &0  & -1 & 1 & -2& -1\\
1  &0  & 2  & 0 & -1& 1\\
0  &0  & 1  & 0 & 2 & 0\\
0  &0  & 0  & 0 & 1 & 0\\
\end{array}
\right]
\left[
    \begin{array}{c}
    -\overline{N}_0\\
    \overline{D}_0\\
    -\overline{N}_1\\
    \overline{D}_1\\
    -\overline{N}_2\\
    \overline{D}_2\\
    \end{array}
\right]=0
\] 
the rank of \textbf{S} is 5, so  there are 2 linearly indepedent N-columns,the degree of the transfer function is 2 .we can also calculate a monic null vector 
\[
z=\left[
    \begin{array}{cccccc}
        -1 & 2 & 0 & 3 & 0 & 1
    \end{array}
\right]'   
\]
\section*{7.5}
just as the problem (7.4) show,in the same way:
\[
\begin{aligned}
\begin{split}
&D(s)=-1+4s^2\\
&N(s)=-1+2s\\
&\overline{D}(s)=\overline{D}_0+\overline{D}_1s\\
&\overline{N}(s)=\overline{N}_0+\overline{N}_1s\\
\end{split}
\end{aligned}
\]
the Sylvester resultant:
\[
\textbf{SM}:=
\left[
    \begin{array}{cccc}
    -1 & -1 & 0 & 0 \\
    0  & 2  & -1& -1 \\
    4  & 0  & 0 & 2 \\
    0  & 0  & 4 & 0
    \end{array}
\right]
\left[
    \begin{array}{c}
        -\overline{N}_0\\
        \overline{D}_0\\
        -\overline{N}_1\\
        \overline{D}_1\\
    \end{array}
\right]=0
\]
the rank of $\textbf{S}$ is 3,we can calculate a monic null vector
\[z=
\left[
    \begin{array}{cccc}
        -0.5 & 0.5 & 0 & 1
    \end{array}
\right]'
\]
so the coprime fraction is:
\[
   \hat{g(s)}=\frac{1}{2s+1}
\]

\section*{7.6}
The Sylvester resultant by arranging the coefficients of $N(s)$ and $D(s)$ in descending powers:
\[
    \left[
        \begin{array}{cccc}
        1 & 0 & 0 & 0\\
        2 & 1 & 1 & 0\\
        0 & 2 & 2 & 1\\
        0 & 0 & 0 & 2\\
        \end{array}
    \right]    
\]
the second D-columns is linearly depedent of its LHS columns, so it is not true that all D-columns are linearly indepedent of their LHS columns.
the degree of $\hat{g(s)}$ is 1,but the linearly indepedent N-columns is 2.

\section*{7.7}
the realization is a controllable realization,the controllable realization is observable if and only if the transfer function is coprime fraction.$D(s)$ and $N(s)$ are coprime if and only if the Sylvester resultant is nonsingular.
\[
\hat{g(s)}=\frac{N(s)}{D(s)}=\frac{\beta_1s+\beta_2}{s^2+\alpha_1s+\alpha_2}    
\]
the Sylvester resultant:
\[
    \textbf{S}:=
    \left[
        \begin{array}{cccc}
            \alpha_2 & \beta_2 & 0 &0\\
            \alpha_1 & \beta_1 & \alpha_2 & \beta_2\\
            1 & 0 &\alpha_1 & \beta_1\\
            0 & 0 & 1& 0
        \end{array}
    \right]
\]
The determinant of \textbf{S} is $-\alpha_2\beta_1^2+\alpha_1\beta_1\beta_2-\beta_2^2$
for the controllable realization, its observability matrix:
\[
O=\left[
    \begin{array}{cc}
        \beta_1 & \beta_2\\
        -\alpha_1\beta_1+\beta_2 & -\alpha_2\beta_1
    \end{array}
\right]    
\]
The realization is observable if and only if the observability matrix is full column rank,the determinant of $O$ is  $-\alpha_2\beta_1^2+\alpha_1\beta_1\beta_2-\beta_2^2$
Thus the two condition is same.
\section*{7.8}
Let us consider a transfer function:
\[
\hat{g(s)}=\frac{N(s)}{D(s)}=\frac{\beta_1s^2+\beta_2s+\beta_3}{s^3+\alpha_1s^2+\alpha_2s+\alpha_3}   
\]
its Sylvester resultant:
\[
    \textbf{S}:=
    \left[
        \begin{array}{cccccc}
            \alpha_3 & \beta_3 & 0        &0        & 0        & 0   \\
            \alpha_2 & \beta_2 & \alpha_3 & \beta_3 & 0        & 0   \\
            \alpha_1 & \beta_1 & \alpha_2 & \beta_2 & \alpha_3 & \beta_3 \\
            1        & 0       & \alpha_1 & \beta_1 & \alpha_2 & \beta_2   \\
            0        & 0       & 1        & 0       & \alpha_1 & \beta_1 \\
            0        & 0       & 0        & 0       & 1        & 0
        \end{array}
    \right]
\]
The observability matrix:
\[
O=
\left[
    \begin{array}{ccc}
    \beta_1 & \beta_2 & \beta_3\\
    -\alpha_1\beta_1+\beta_2 & -\alpha_2\beta_1+\beta_3 & -\alpha_3\beta_1\\
    (\alpha_1^2-\alpha_2)\beta_1-\alpha_1\beta_2+\beta_3 &(\alpha_1\alpha_2-\alpha_3)\beta_1-\alpha_2\beta_2 & \alpha_1\alpha_3\beta_1-\alpha_3\beta_2
    \end{array}
\right]    
\]

\section*{7.9}
its controllable realization is:
\[
\dot{x}=
\left[
\begin{array}{cc}
-2 & -1\\
1 & 0\\
\end{array}    
\right]x+
\left[
    \begin{array}{c}
    1\\
    0
    \end{array}
\right]u
\]
\[
    y=\left[
        \begin{array}{cc}
            0 & 1
        \end{array}
        \right]x    
\]
its observability matrix is:
\[
    O=\left[
        \begin{array}{cc}
            0 & 1\\
            1 & 0
        \end{array}
    \right]      
\]
its determinant is nonzero, so the realization is observable.
The Sylvester resultant of D(s) and N(S) is:
\[
    \textbf{S}:=
    \left[
        \begin{array}{cccc}
            1 & 1 & 0 &0\\
            2 & 0 & 1 & 1\\
            1 & 0 &2 & 0\\
            0 & 0 & 1& 0
        \end{array}
    \right]
\]
its determinant is also nonzero,so the Sylvester resultant is nonsingular.

\section*{7.10}
\[
\hat{g(s)}=\frac{1}{(s+1)^2}=s^{-2}-2s^{-3}+3s^3
\]
the irreducible companion form:
\[
\dot{x}=
\left[
\begin{array}{cc}
0 & 1\\
-1 & -2\\
\end{array}    
\right]x+
\left[
    \begin{array}{c}
    0\\
    1
    \end{array}
\right]u
\]
\[
    y=\left[
        \begin{array}{cc}
            1 & 0
        \end{array}
        \right]x    
\]
\section*{7.11}
from the problem (7.10)
\[T(2,2)=
\left[
    \begin{array}{cc}
        0 & 1\\
        1 & -2
    \end{array}
\right]
\]
\[
\tilde{T}(2,2)=OAC=
\left[
    \begin{array}{cc}
        1 & -2\\
        -2 & 3\\        
    \end{array}
\right]  
\]
using the singular value decomposition to express $T(2,2)$ as
\[T(2,2)=OC=K\wedge L'\] 
where $K$ and $L'$ are orthogonal matrix.
Let $O=K\wedge^{\frac{1}{2}}$ and $C=\wedge^{\frac{1}{2}}L'$
\[
\begin{split}
O=\left[
    \begin{array}{cc}
        0.5946 & -0.5946\\
        -1.4355 & -0.2463\\
    \end{array}
\right]
C=\left[
    \begin{array}{cc}
        -0.5946 & -1.4355\\
        0.5946 & -0.2463\\
    \end{array}
\right]
\end{split} 
\]
\[
\begin{split}
&A=O^{-1}\tilde{T}(2,2)C^{-1}=
\left[
    \begin{array}{cc}
        1.000 & 3.4142\\
        -0.5858 & -1.000\\
    \end{array}
\right]
\\
&b=
\left[
    \begin{array}{c}
        -0.5946\\
        0.5946\\
    \end{array}
\right]
\\
&c=\left[
    \begin{array}{cc}
        0.5946 & -0.5946
    \end{array}
\right]
\end{split}
\]
\[
\dot{x}=
\left[
    \begin{array}{cc}
        1.000 & 3.4142\\
        -0.5858 & -1.000\\
    \end{array}
\right]x+
\left[
    \begin{array}{c}
        -0.5946\\
        0.5946\\
    \end{array}
\right]u
\]
\[
    y=\left[
        \begin{array}{cc}
            0.5946 & -0.5946
        \end{array}
    \right]x    
\]

\section*{7.12}

\[
\hat{g(s)}=\frac{2s+2}{s^2-s-2}=\frac{2}{s-2}    
\]
we can see that the transfer function can be reduced to a coprime fucntion withd degree 1.
so they are not minimal realizations,and they are not algebraically equivalent,because they have differnert eigenvalues.

\section*{7.13}
The character polynomials of $\hat{G_1(s)}$ is $s(s+1)(s+3)$ and the degree is 3.\\
The character polynomials of $\hat{G_2(s)}$ is $(s+1)^3(s+2)^2$ and the degree is 5.\\
The character polynomials of $\hat{G_2(s)}$ is $s(s+1)^(s+2)(s+3)^2(s+4)(s+5)$ and the degree is 8.\\

\section*{7.14}
\[\hat{G(s)}=
\left[
    \begin{array}{cc}
        s & 1\\
        -s & s
    \end{array}
\right]^{-1}
\left[
    \begin{array}{c}
        1\\
        -1
    \end{array}
\right]
\]
\[
\overline{D}(s)=
\left[
    \begin{array}{cc}
        s & 1\\
        -s & s
    \end{array}
\right]=
\left[
    \begin{array}{cc}
        0 & 1\\
        0 & 0\\
    \end{array}
\right]+
\left[
    \begin{array}{cc}
        1 & 0\\
        -1 & 1
    \end{array}
\right]s
\]
\[
\overline{N}(s)=\left[
    \begin{array}{c}
        1\\
        -1
    \end{array}
\right]
\]
The generalized resultant:
\[
\textbf{SM}:=
\left[
    \begin{array}{cccccc}
    0 & 1 & 1 & 0 & 0 & 0 \\
    0  & 0  & -1& 0 & 0 & 0 \\
    1  & 0  & 0 & 0 & 1 & 1 \\
    -1  & 1  & 0 & 0 & 0 & -1\\
    0  & 0 & 0 & 1 & 0 &0 \\
    0 & 0 &0 & -1 & 1 & 0\\
    \end{array}
\right]
\left[
    \begin{array}{c}
        -\overline{N}_0\\
        \overline{D}_0\\
        -\overline{N}_1\\
        \overline{D}_1\\
    \end{array}
\right]=0
\]
The nullity of $S$ is 
\[
\left[
    \begin{array}{c}
        -1\\
        0\\
        0\\
        0 \\
        0\\
        1\\
    \end{array}
\right]    
\]
so we can get:
\[
N_0=\left[
    \begin{array}{c}
        -1\\
        0
    \end{array}
\right]    
D_0=0
N_1=\left[
    \begin{array}{c}
        0\\
        0
    \end{array}
\right]
D_1=1   
\]
so we have
\[
    \hat{G(s)}=
    \left[
        \begin{array}{c}
          1\\
          0
        \end{array}
    \right]s^{-1}   
\]
The degree of $G(s)$ is 1,deg det $\overline{D}(s)=2$,so the left fraction is not coprime.

\section*{7.15}
we form from 
\[\hat{G(s)}=
\left[
    \begin{array}{cc}
        s & 1\\
        -s & s
    \end{array}
\right]^{-1}
\left[
    \begin{array}{c}
        1\\
        -1
    \end{array}
\right]
\]
 get 
\[
\left[
    \begin{array}{cccccc}
    1 & 0& 0 & 0 & 0 & 0 \\
    -1  & 1  & 0& 0 & 0 & 0 \\
    0  & 1  & 1 & 1 & 0 & 0 \\
    0  & 0  & -1 & -1 & 1 & 0\\
    0  & 0 & 0 & 0 & 1 &1 \\
    0 & 0 &0 & 0 & 0 & -1\\
    \end{array}
\right]
\]
where $D(s)$ and $N(s)$ are arranged in descending powers of s,it is not true that all D-columns are linearly indepedent of their LHS columns, bescuse the second block of $D_1$-columns is linearly depedent of its LHS columns.
The number of linearly indepedent N-columns is 2,so the degree is not equal the number of linearly indepedent N-columns,so the theorem is not hold either.

\section*{7.16}
\[
    \hat{G(s)}=
    \left[
        \begin{array}{c}
          1\\
          0
        \end{array}
    \right]s^{-1}   
\]

The generalized resultant
\[
T:=\left[
    \begin{array}{ccc}
        0 & 1 & 0\\
        1 & 0 & 0\\
        0 & 0 & 0\\
        0 & 0 & 1\\
        0 & 1 & 0\\
        0 & 0 & 0\\
    \end{array}
\right]    
\]
N2-row in the first N block-row is linearly depedent of its preceding rows,
Let 
\[
t1:=\left[
    \begin{array}{ccc}
        0 & 1 & 0\\
        1 & 0 & 0\\
        0 & 0 & 0\\
        0 & 0 & 0\\
        0 & 0 & 0\\
        0 & 0 & 0\\
    \end{array}
\right]    
\]
\[
    \left[
        \begin{array}{cccccc}
            0 & 0 & 1 & 0 & 0 &0\\
        \end{array}
    \right]t1=0    
\]
N1-row in the second block-row is linearly depedent of its preceding rows,let(deleting the N2-row in the first block)
\[
t2:=\left[
    \begin{array}{ccc}
        0 & 1 & 0\\
        1 & 0 & 0\\
        0 & 0 & 1\\
        0 & 1 & 0\\
        0 & 0 & 0\\
    \end{array}
\right]
\]  
\[
    \left[
        \begin{array}{cccccc}
            -1 & 0 & 0 & 1 & 0 &0\\
        \end{array}
    \right]t2=0    
\]
\[
\left[
    \begin{array}{cccc}
        -\overline{N}_0 & \overline{D}_0 & -\overline{N}_1 & \overline{D}_1
    \end{array}
\right]=
\left[
    \begin{array}{cccccc}
        0 & 0 & 1 & 0 & 0 &0\\
        -1 & 0 & 0 & 0 & 1 &0\\
    \end{array}
\right]
\]
we can get 
\[
    \begin{split}
    &\overline{N}(s)=\overline{N}_0+\overline{N}_1s\\
    &\overline{D}(s)=\overline{D}_0+\overline{D}_1s
    \end{split}
\]
\[
\hat{G}(s)=
\left[
    \begin{array}{cc}
        0 & 1\\
        s & 0
    \end{array}
\right]^{-1}
\left[
    \begin{array}{c}
        0\\
        1
    \end{array}
\right]   
\]


\section*{7.17}
\[
\begin{array}{l}
 \hat{G}(s)=\left[\begin{array}{cc}
    \frac{s^{2}+1}{s^{3}} & \frac{2 s+1}{s^{2}} \\
    \frac{s+2}{s^{2}} & \frac{2}{s}
    \end{array}\right]=: \bar{D}^{-1}(S) \bar{N}(S) \\
    \bar{D}(s)=\left[\begin{array}{cc}
    s^{3} & 0 \\
    0 & s^{2}
    \end{array}\right]=\left[\begin{array}{cc}
    0 & 0 \\
    0 & 0
    \end{array}\right]+\left[\begin{array}{cc}
    0 & 0 \\
    0 & 0
    \end{array}\right] s+\left[\begin{array}{cc}
    0 & 0 \\
    0 & 1
    \end{array}\right] s^{2}+\left[\begin{array}{cc}
    1 & 0 \\
    0 & 0
    \end{array}\right] s^{3} \\
    \text { where. } \\
    \bar{N}(s)=\left[\begin{array}{cc}
    s^{2}+1 & s(2 s+1) \\
    s+2 & 2 s
    \end{array}\right]=\left[\begin{array}{cc}
    1 & 0 \\
    2 & 0
    \end{array}\right]+\left[\begin{array}{cc}
    0 & 1 \\
    1 & 2
    \end{array}\right] s+\left[\begin{array}{cc}
    1 & 2 \\
    0 & 0
    \end{array}\right] s^{2}
    \end{array}
\]

the generalized resultant-is
\[
S=\left[\begin{array}{llllllllllll}
0 & 0 & 1 & 0 & 0 & 0 & 0 & 0 & 0 & 0 & 0 & 0 \\
0 & 0 & 2 & 0 & 0 & 0 & 0 & 0 & 0 & 0 & 0 & 0 \\
0 & 0 & 0 & 1 & 0 & 0 & 1 & & 0 & 0 & 0 & 0 \\
0 & 0 & 1 & 2 & 0 & 0 & 2 & 0 & 0 & 0 & 0 & 0 \\
0 & 0 & 1 & 2 & 0 & 0 & 0 & 1 & 0 & 0 & 1 & 0 \\
0 & 1 & 0 & 0 & 0 & 0 & 1 & 2 & 0 & 0 & 2 & 0 \\
1 & 0 & 0 & 0 & 0 & 0 & 1 & 2 & 0 & 0 & 0 & 1 \\
0 & 0 & 0 & 0 & 0 & 1 & 0 & 0 & 0 & 0 & 1 & 2 \\
0 & 0 & 0 & 0 & 1 & 0 & 0 & 0 & 0 & 0 & 1 & 2 \\
0 & 0 & 0 & 0 & 0 & 0 & 0 & 0 & 0 & 1 & 0 & 0 \\
0 & 0 & 0 & 0 & 0 & 0 & 0 & 0 & 1 & 0 & 0 & 0 \\
0 & 0 & 0 & 0 & 0 & 0 & 0 & 0 & 0 & 0 & 0 & 0
\end{array}\right]
\]
$\operatorname{rank} s=9, \quad \mu_{1}=2, \mu_{2}=1$
the monic null vectors of the submatrices that consist of the primary dependent- $\bar{N}_{2}$ -columns and
$\bar{N}_{1}$ -columns are , respectively
$Z 2=\left[\begin{array}{llllllll}-2.5 & -2.5 & 0 & -0.5 & 0 & 0 & 0.5 & 1\end{array}\right]$
$Z 1=\left[\begin{array}{llllllllll}-0.5 & -2.5 & 0 & -0.5 & -1 & -1 & 0.5 & 0 & 0 & 1\end{array}\right]$
\[
\therefore\left[\begin{array}{c}
-N_{0} \\
D_{0} \\
-N_{1} \\
D_{1} \\
N_{2} \\
D_{2}
\end{array}\right]=\left[\begin{array}{cccccccccccc}
-0.5 & -2.5 & 0 & -0.5 & -1 & -1 & 0.5 & 0 & 0 & 0 & 1 & 0 \\
-2.5 & -2.5 & 0 & -0.5 & 0 & 0 & 0.5 & 1 & 0 & 0 & 0 & 0
\end{array}\right]
\]
\[
\begin{aligned}
    &D(s)=\left[\begin{array}{cc}
    0 & 0 \\
    -0.5 & -0.5
    \end{array}\right]+\left[\begin{array}{cc}
    0.5 & 0.5 \\
    0 & 1
    \end{array}\right] s+\left[\begin{array}{cc}
    1 & 0 \\
    0 & 0
    \end{array}\right] s^{2}=\left[\begin{array}{cc}
    s^{2}+0.5 s & 0.5 s \\
    -0.5 & s-0.5
    \end{array}\right]\\
    &N(s)=\left[\begin{array}{ll}
    0.5 & 2.5 \\
    2.5 & 2.5
    \end{array}\right]+\left[\begin{array}{ll}
    1 & 0 \\
    1 & 0
    \end{array}\right] s=\left[\begin{array}{ll}
    s+0.5 & 2.5 \\
    s+2.5 & 2.5
    \end{array}\right]\\
    &\underline{\text { thus }} \text { - right coprime fraction of } \hat{G}(s) \cdot \text { is } \hat{G}(s)=\left[\begin{array}{ll}
    s+0.5 & 2.5 \\
    s+2.5 & 2.5
    \end{array}\right]\left[\begin{array}{cc}
    s^{2}+0.5 s & 0.5 s \\
    -0.5 & s-0.5
    \end{array}\right]\\
    &\text { we define } H(s)=\left[\begin{array}{cc}
    s^{2} & 0 \\
    0 & s
    \end{array}\right] \quad L(s)=\left[\begin{array}{cc}
    s & 0 \\
    1 & 0 \\
    0 & 1
    \end{array}\right]
    \end{aligned}
    \]

    then -we have
    $$
    \begin{array}{l}
    D(s)=\left[\begin{array}{cc}
    1 & 0.5 \\
    0 & 1
    \end{array}\right] H(s)+\left[\begin{array}{ccc}
    0.5 & 0 & 0 \\
    0 & -0.5 & -0.5
    \end{array}\right] L(s) \\
    N(s)=\left[\begin{array}{ccc}
    1 & 0.5 & 2.5 \\
    1 & 2.5 & 2.5
    \end{array}\right] L(s) \\
    D_{h c}^{-1}=\left[\begin{array}{cc}
    1 & 0.5 \\
    0 & 1
    \end{array}\right]^{-1}=\left[\begin{array}{cc}
    1 & -0.5 \\
    0 & 1
    \end{array}\right]\\
    D_{hc}^{-1} D_{lc}=\left[\begin{array}{cc}
    1 & -0.5 \\
    0 & 1
    \end{array}\right]\left[\begin{array}{ccc}
    0.5 & 0 & 0 \\
    0 & -0.5 & -0.5
    \end{array}\right]=\left[\begin{array}{ccc}
    0.5 & 0.25 & 0.25 \\
    0 & -0.5 & -0.5
    \end{array}\right]
    \end{array}
    $$
    thus - minimal realization of $\hat{G}(s)$ is
    $$
    \dot{x}=\left[\begin{array}{ccc}
    -0.5 & -0.25 & -0.25 \\
    1 & 0 & 0 \\
    0 & 0.5 & 0.5
    \end{array}\right] x+\left[\begin{array}{cc}
    1 & -0.5 \\
    0 & 0 \\
    0 & 1
    \end{array}\right] u
    $$
    $$
    y=\left[\begin{array}{lll}
    1 & 0.5 & 2.5 \\
    1 & 2.5 & 2.5
    \end{array}\right]x
    $$
\end{document}