\documentclass{article}
% 这里是导言区
%\usepackage{indentfirst}%缩进控制
\usepackage{listings}%插入代码
%\usepackage{mcode}
%ctex能够保证能够渲染英文
\usepackage{ctex}
\usepackage{textcomp}
\usepackage{graphicx}%插入图像
\usepackage{epstopdf}
\usepackage{amsmath}
\usepackage{graphicx}
\usepackage{subfigure}
\usepackage{geometry}%设置页边距
\usepackage{amssymb}
\usepackage{float}
\usepackage[level]{datetime} 
\makeatletter
\newcommand{\rmnum}[1]{\romannumeral #1}
\newcommand{\Rmnum}[1]{\expandafter\@slowromancap\romannumeral #1@}
\makeatother
% \renewcommand\thesection{\roman{subsection}}
%\newdateformat{ukdate}{\ordinaldate{\THEDAY} \monthname[\THEMONTH]

\geometry{a4paper,scale=0.75}


\lstset{
tabsize=4, %tab 空格数
frame=shadowbox, %把代码用带有阴影的框圈起来
rulesepcolor=\color{red!20!green!20!blue!20}, %代码块边框为淡青色
keywordstyle=\color{blue!90}\bfseries, %代码关键字的颜色为蓝色, 粗体
showstringspaces=false, %不显示代码字符串中间的空格标记
stringstyle=\ttfamily, %代码字符串的特殊格式
keepspaces=true, %
breakindent=22pt, %
numbers=left, %左侧显示行号
stepnumber=1, %
numberstyle=\tiny, %行号字体用小号
basicstyle=\footnotesize, %
showspaces=false, %
flexiblecolumns=true, %
breaklines=true, %对过长的代码自动换行
breakautoindent=true, %
breakindent=4em, %
aboveskip=1em, %代码块边框
}

\title{Chapter4}
\author{31202008881        \quad \quad \quad
          Bao Ze an}

\begin{document}
\setlength{\parindent}{2em}
\maketitle

\section*{4.1}
The system is a LTI system,so the solution can be obtained by:$\mathcal{L}^{-1}(SI-A)^{-1}$
\[(SI-A)^{-1}=
\left[
\begin{array}{cc}
S &-1\\
1 & S\\
\end{array}
\right]^{-1}
=
\left[
\begin{array}{cc}
\frac{s}{s^2+1} & \frac{1}{s^2+1}\\
\frac{-1}{s^2+1} & \frac{s}{s^2+1}\\
\end{array}
\right]
\]
\[\mathcal{L}^{-1}(SI-A)^{-1}=
\left[
\begin{array}{cc}
cost & sint\\
-sint & cost\\
\end{array}
\right]
\]
so the solution is:
\[
x(t)=
\left[
\begin{array}{cc}
cost & sint\\
-sint & cost\\
\end{array}
\right]
x(0)  
\]

\section*{4.2}
The first method: $G(s)=C(SI-A)^{-1}B+D$
\[G(s)=
\left[
\begin{array}{cc}
2 & 3
\end{array}
\right]
\left[
\begin{array}{cc}
\frac{s+2}{s^2+s2+2}& \frac{1}{s^2+2s+2} \\
\frac{-2}{s^2+2s+2} & \frac{s}{s^2+2s+2}\\
\end{array}
\right]
\left[
\begin{array}{c}
1\\
1\\
\end{array}
\right]
=
\frac{5s}{s^2+2s+2}
\]
so the unit-step response is:$y(s)=G(s)u(s)=\frac{5s}{s^2+2s+2}.\frac{1}{s}=\frac{5}{(s+1)^2+1}$
\[y(t)=\mathcal{L}^{-1}(y(s))=5e^{-t}sint\]
The second method:calculate the solution of $x(t)$,then we can simply get the response by $y(t)=Cx(t)+Du(t)$
\[x(t)=e^{At}x(0)+\int_{0}^{t}e^{A(t-\tau)}Bu(\tau)d\tau\]
because of zero initial state,so:
\[y(t)=C\int_{0}^{t}e^{A(t-\tau)}Bu(\tau)d\tau\]
from the first method:
\[
(SI-A)^{-1}
\left[
\begin{array}{cc}
\frac{s+2}{s^2+s2+2}&\frac{1}{s^2+2s+2} \\
\frac{-2}{s^2+2s+2} & \frac{s}{s^2+2s+2}\\
\end{array}
\right]
\]
\[e^{At}
=
\left[
\begin{array}{cc}
e^{-t}(cost+sint) & e^{-t}sint \\
-2e^{-t}sint & e^{-t}(cost-sint)\\
\end{array}
\right]
\]
take a substitution,we can get:
\[y(t)=\int_{0}^{t}5e^{-(t-\tau)}[cos(t-\tau)-sin(t-\tau)]d\tau=5e^{-t}sint\]

\section*{4.3}
Let $u(t)$ is piecewise constant,that is to say,the input changes values only at discrete-time instants.
we get the discrete-time equation without the approximation:
\[
\begin{aligned}
x[k+1]=A_dx[k]+B_du[k]\\
y[k]=C_dx[k]+D_du[k]\\
\end{aligned}    
\]
where $A_d=e^{AT},B_d=(\int_{0}^{T}e^{A\tau}d\tau)B,C_d=C,D_d=D$
when $T=1$:
\[
A_d=e^{A}=
\left[
\begin{array}{cc}
e^{-1}(cos1+sin1) & e^{-1}sin1 \\
-2e^{-1}sin1 & e^{-1}(cos1-sin1)\\
\end{array}
\right]
\]
%不要老是用calculate表示计算这个英文,要学会使用compute这个单词
Because A is nonsingular, so we can compute the
\[
B_d=A^{-1}(A_d-I)B=
\left[
\begin{array}{c}
1.0491\\
-0.1821
\end{array}
\right]    
\]
\[
C_d=C=
\left[
\begin{array}{cc}
2 &3\\
\end{array}
\right] 
\]
thus the discrete-time equation:
\[
x[k+1]=
\left[
\begin{array}{cc}
e^{-1}(cos1+sin1) & e^{-1}sin1 \\
-2e^{-1}sin1 & e^{-1}(cos1-sin1)\\
\end{array}
\right]x[k]+
\left[
\begin{array}{c}
1.0491\\
-0.1821
\end{array}
\right]u[k]
\]
\[
y[k]=
\left[
\begin{array}{cc}
2 &3\\
\end{array}
\right]x[k]
\]
in the same way,for $T=\pi$:
\[
x[k+1]=
\left[
\begin{array}{cc}
-0.0432 & 0 \\
0 & -0.0432\\
\end{array}
\right]x[k]+
\left[
\begin{array}{c}
1.5648\\
-1.0432
\end{array}
\right]u[k]
\]
\[
y[k]=
\left[
\begin{array}{cc}
2 &3\\
\end{array}
\right]x[k]
\]

\section*{4.4}
%求标准型的方法不只是有求传递函数,三阶有的时候(SI-A)逆很难求,
%这个时候就要想起等价变换去求
For companion form,we can compute the tansformation matrix:
\[Q=
\left[
\begin{array}{ccc}
b & Ab & A^2b
\end{array}
\right]
=
\left[
\begin{array}{ccc}
1 & -2 & 4\\
0 & 2 & -4\\
1 & -2 & 0
\end{array}
\right]
\]
\[
\dot{\overline{x}}=Q^{-1}AQ\overline{x}+Q^{-1}bu
\]
\[y=CQ\overline{x}\]
substitue with $Q,A,B,C$,we can get:
\[
\dot{\overline{x}}=
\left[
\begin{array}{ccc}
0 & 0 & -4\\
1 & 0 & -6\\
0& 1 & -4
\end{array}
\right]\overline{x}+
\left[
\begin{array}{c}
1\\
0\\
0\\
\end{array}
\right]
\]
\[y=
\left[
\begin{array}{ccc}
1 & -4 & 8
\end{array}
\right]\overline{x}
\]
For modal form:
The characteristic polynomial of $A$,
\[det(SI-A)=(s+2)(s^2+2s+2)\]
The eigenvalues of $A$ are:-2,-1+j,-1-j,
their corresponding eigenvectors are $q_1,q_2,q_3$
\[
q_1=    
\left[
\begin{array}{c}
0.7071\\
0\\
-0.7071
\end{array}
\right]
\quad
q_2=    
\left[
\begin{array}{c}
0\\
0.5774j\\
-0.5774-0.5774j
\end{array}
\right]
\quad
q_3=    
\left[
\begin{array}{c}
0\\
-0.5774j\\
-0.5774+0.5774j
\end{array}
\right]
\]
so,the transformation matrix is:
\[
Q=
\left[
\begin{array}{ccc}
q_1 & Re(q_2) &Im(q_2)
\end{array}
\right]=
\left[
\begin{array}{ccc}
0.7071 & 0 &0\\
0 &0 &0.5774\\
-0.7071 &-0.5774 & -0.5774 
\end{array}
\right]    
\]
we can get the modal form by:
\[
\dot{\overline{x}}=Q^{-1}AQ\overline{x}+Q^{-1}bu
\]
\[y=CQ\overline{x}\]
thus:
\[
\dot{\overline{x}}=
\left[
\begin{array}{ccc}
-2 & 0 & 0\\
0 & -1 & 1\\
0& -1 & -1
\end{array}
\right]\overline{x}+
\left[
\begin{array}{c}
-3.4638\\
0\\
1.4138\\
\end{array}
\right]
\]
\[y=
\left[
\begin{array}{ccc}
0 & -0.5774 & 0.7071
\end{array}
\right]\overline{x}
\]
\section*{4.5}
For unit step input ,we can get $|y|_{max}=0.55,|x_{1}|_{max}=0.5,|x_{2}|_{max}=1.05,|x_{3}|_{max}=0.52$. Define $\bar{x}_{1}=x_{1},\bar{x}_{2}=0.5x_{2},\bar{x}_{3}=x_{3}$, so
$$
P=
\begin{bmatrix}
1&0&0\\
0&0.5&0\\
0&0&1
\end{bmatrix}
$$$$
P^{-1}=
\begin{bmatrix}
1&0&0\\
0&2&0\\
0&0&1
\end{bmatrix}
$$
$$
\begin{aligned}
\dot{\bar{\pmb{x}}}&=PAP^{-1}\bar{\pmb{x}}+Pbu\\
&=\begin{bmatrix}
-2&0 &0\\
0.5&0&0.5\\
0&-4&-2
\end{bmatrix}\bar{\pmb{x}}+\begin{bmatrix}
1\\0\\1
\end{bmatrix}u\\
y&=cP^{-1}\bar{\pmb{x}}=\begin{bmatrix}
1&-2&0\\
\end{bmatrix}\bar{\pmb{x}}
\end{aligned}
$$
The largest permissible a is $\frac{10}{0.55}=18.2$
\section*{4.6}
$$
\begin{array}{l}
\dot{\bar{\pmb{x}}}=Q^{-1}\left[\begin{array}{cc}
\lambda & 0 \\
0 & \bar{\lambda}
\end{array}\right] Q \bar{\pmb{x}}+Q^{-1}\left[\begin{array}{c}
b_{1} \\
b_{2}
\end{array}\right] u \\
=\frac{1}{(\lambda-\bar{\lambda}) b_{1} \bar{b}_{1}}\left[\begin{array}{cc}
\bar{b}_{1} & -b_{1} \\
\lambda \bar{b}_{1} & -\bar{\lambda} b_{1}
\end{array}\right]\left[\begin{array}{cc}
\lambda & 0 \\
0 & \bar{\lambda}
\end{array}\right]\left[\begin{array}{cc}
-\bar{\lambda} b_{1} & b_{1} \\
-\lambda \bar{b}_{1} & \bar{b}_{1}
\end{array}\right] \bar{\pmb{x}}+\frac{1}{(\lambda-\bar{\lambda}) b_{1} \bar{b}_{1}}\left[\begin{array}{cc}
\bar{b}_{1} & -b_{1} \\
\lambda \bar{b}_{1} & -\bar{\lambda} b_{1}
\end{array}\right]\left[\begin{array}{c}
b_{1} \\
b_{2}
\end{array}\right] u \\
=\frac{1}{(\lambda-\bar{\lambda}) b_{1} \bar{b}_{1}}\left[\begin{array}{cc}
\bar{b}_{1} \lambda & -b_{1} \bar{\lambda} \\
\lambda^{2} \bar{b}_{1} & -\bar{\lambda}^{2} b_{1}
\end{array}\right]\left[\begin{array}{cc}
-\bar{\lambda} b_{1} & b_{1} \\
-\lambda \bar{b}_{1} & \bar{b}_{1}
\end{array}\right] \bar{\pmb{x}}+\frac{1}{(\lambda-\bar{\lambda}) b_{1} \bar{b}_{1}}\left[\begin{array}{c}
0 \\
(\lambda-\bar{\lambda}) b_{1} \bar{b}_{1}
\end{array}\right] u \\
=\left[\begin{array}{cc}
0 & 1 \\
-\lambda \bar{\lambda} & \lambda+\bar{\lambda}
\end{array}\right] \bar{\pmb{x}}+\left[\begin{array}{c}
0 \\
1
\end{array}\right] u=\bar{A} \bar{\pmb{x}}+\bar{B} u \\
y=\left[\begin{array}{cc}
c_{1} & \bar{c}_{1}
\end{array}\right] Q \bar{\pmb{x}}=\left[\begin{array}{cc}
c_{1} & \bar{c}_{1}
\end{array}\right]\left[\begin{array}{cc}
-\bar{\lambda} b_{1} & b_{1} \\
-\lambda \bar{b}_{1} & \bar{b}_{1}
\end{array}\right] \bar{\pmb{x}}=\left[\begin{array}{cc}
-\bar{\lambda} b_{1} c_{1}-\lambda \bar{b}_{1} \bar{c}_{1} & c_{1} b_{1}+\bar{c}_{1} \bar{b}_{1}
\end{array}\right] \bar{\pmb{x}}=\bar{C}_{1} \bar{\pmb{x}}
\end{array}
$$
\section*{4.7}
Change the order of the state variables from $\begin{bmatrix}
\dot{x_{1}} &
\dot{x_{2}} &
\dot{x_{3}}&
\dot{x_{4}} &
\dot{x_{5}} &
\dot{x_{6}}\end{bmatrix}^{'}$ to $\begin{bmatrix}\dot{x_{1}} &
\dot{x_{4}} &
\dot{x_{2}} &
\dot{x_{5}}&
\dot{x_{3}} &
\dot{x_{6}}\end{bmatrix}^{'}$ we can get
$$
\begin{aligned}
\dot{\bar{\pmb{x}}}&
=\begin{bmatrix}
\dot{x_{1}} \\
\dot{x_{4}} \\
\dot{x_{2}} \\
\dot{x_{5}}\\
\dot{x_{3}} \\
\dot{x_{6}}
\end{bmatrix}=\begin{bmatrix}
\lambda && 1  & & \\
& \bar{\lambda} & & 1 & & \\
& & \lambda & & 1 & \\
& & & \bar{\lambda} & & 1 \\
& & & & \lambda & \\
& & & & & \bar{\lambda}
\end{bmatrix}
\begin{bmatrix}
x_{1} \\
x_{4} \\
x_{2} \\
x_{5} \\
x_{3} \\
x_{6}
\end{bmatrix}+\begin{bmatrix}
b_{1} \\
\bar{b}_{1} \\
b_{2} \\
\bar{b}_{2} \\
b_{3} \\
\bar{b}_{3}
\end{bmatrix}=\begin{bmatrix}
\bar{A} & I_{2} & 0 \\
0 & \bar{A} & I_{2} \\
0 & 0 & \bar{A}
\end{bmatrix}\bar{\pmb{x}}+\begin{bmatrix}
\bar{B}_{1} \\
\bar{B}_{2} \\
\bar{B}_{3}
\end{bmatrix} u\\
y&=\left[\begin{array}{llllll}
c_{1} & \bar{c}_{1} & c_{2} & \bar{c}_{2} & c_{3} & \bar{c}_{3}
\end{array}\right] \bar{\pmb{x}}=\left[\begin{array}{lll}
\bar{C}_{1} & \bar{C}_{2} & \bar{C}_{3}
\end{array}\right] \bar{\pmb{x}}\end{aligned}
$$
\section*{4.8}
The eigenvalues of the first equation are $2,2,1$, and eigenvalues of the second equation are $2,2,-1$. So they are not equivalent.
$$
\begin{aligned}
&\hat{G}_{1}(s)=C_{1}(s I-A_{1})^{-1} B_{1}=\left[\begin{array}{lll}
1 & -1 & 0
\end{array}\right]\left[\begin{array}{ccc}
s-2 & -1 & -2 \\
0 & s-2 & -2 \\
0 & 0 & s-1
\end{array}\right]^{-1}\left[\begin{array}{l}
1 \\
1 \\
0
\end{array}\right]\\
&=\frac{\left[\begin{array}{ccc}
1 & -1 & 0
\end{array}\right]}{(s-2)^{2}(s-1)}\left[\begin{array}{ccc}
(s-2)(s-1) & (s-1) & 2(s-1) \\
0 & (s-2)(s-1) & 2(s-2) \\
0 & 0 & (s-2)^{2}
\end{array}\right]\left[\begin{array}{l}
1 \\
1 \\
0
\end{array}\right]\\
&=\frac{1}{(s-2)^{2}}\\
&\hat{G}_{2}(s)=C_{2}(s I-A_{2})^{-1} B_{2}=\left[\begin{array}{lll}
1 & -1 & 0
\end{array}\right]\left[\begin{array}{ccc}
s-2 & -1 & -1 \\
0 & s-2 & -1 \\
0 & 0 & s+1
\end{array}\right]^{-1}\left[\begin{array}{l}
1 \\
1 \\
0
\end{array}\right]\\
&=\frac{\left[\begin{array}{ccc}
1 & -1 & 0
\end{array}\right]}{(s-2)^{2}(s+1)}\left[\begin{array}{ccc}
(s-2)(s+1) & (s+1) & (s-1) \\
0 & (s-2)(s+1) & (s-2) \\
0 & 0 & (s-2)^{2}
\end{array}\right]\left[\begin{array}{l}
1 \\
1 \\
0
\end{array}\right]\\
&=\frac{1}{(s-2)^{2}}
\end{aligned}
$$
$\hat{G}_{1}(s)=\hat{G}_{2}(s)$, so they are zero-state equivalent.
\section*{4.9}
Let us define
\[Z:=\begin{bmatrix}Z_{1}&Z_{2}&\cdots&Z_{r}\end{bmatrix}:=C(sI-A)^{-1}\]
where $Z_{i}$ is $q\times q$ and $Z$ is $q\times rq$. Then the transfer matrix of (4.34) equals
\[C(sI-A)^{-1}B=N_{1}Z_{1}+N_{2}Z_{2}+\cdots +N_{r}Z_{r}\] 
\[Z=C(sI-A)^{-1}\quad \Rightarrow \quad ZA=sZ-C\]
Then
$$
\begin{aligned}
sZ-I_{q}&=-\alpha_{1}Z_{1}-\alpha_{2}Z_{2}-\cdots--\alpha_{r}Z_{r}\\
Z_{1}&=sZ_{2} \quad \Rightarrow \quad Z_{2}=\frac{1}{s}Z_{1}\\
Z_{2}&=sZ_{3} \quad \Rightarrow \quad Z_{3}=\frac{1}{s^{2}}Z_{1}\\
&\hspace{1.6cm} \vdots\\
Z_{r-1}&=sZ_{r} \quad \Rightarrow \quad Z_{r}=\frac{1}{s^{r-1}}Z_{1}
\end{aligned}
$$
Then
$$
\begin{aligned}
sZ_{1}-I_{q}&=(-\alpha_{1}-\frac{1}{s}\alpha_{2}-\cdots -\frac{1}{s^{r-1}}\alpha_{r})Z_{1}\\
&\hspace{1.5cm} \Downarrow\\
Z_{1}=\frac{s^{r-1}}{d(s)}&I_{q} \quad Z_{2}=\frac{s^{r-2}}{d(s)}I_{q} \quad
Z_{r}=\frac{1}{d(s)}I_{q} \\
&\hspace{1.5cm} \Downarrow\\
C(sI-A)^{-1}B&=N_{1}Z_{1}+N_{2}Z_{2}+\cdots +N_{r}Z_{r}\\
&=\frac{1}{d(s)}
\begin{bmatrix}N_{1}s^{r-1}&N_{2}s^{r-2}&\cdots&N_{r}\end{bmatrix}\\
&=\hat{G}_{sp}(s)
\end{aligned} 
$$
So this is a realization of $\hat{G}_{sp}(s)$.
\section*{4.10}
In this case,$r=4,q=1$, then
$$
\begin{aligned}
I_{q}&=1\\
N_{1}&=\begin{bmatrix}\beta_{11} &\beta_{12}\end{bmatrix}\\
N_{2}&=\begin{bmatrix}\beta_{21} &\beta_{22}\end{bmatrix}\\
N_{3}&=\begin{bmatrix}\beta_{31} &\beta_{32}\end{bmatrix}\\
N_{4}&=\begin{bmatrix}\beta_{41} &\beta_{42}\end{bmatrix}\end{aligned}
$$
We can get the conclusion.
\section*{4.11}
$$
\begin{aligned}
\hat{G}(s)=\begin{bmatrix}
\frac{2}{s+1} & \frac{2 s-3}{(s+1)(s+2)} \\
\frac{s-2}{s+1} & \frac{s}{s+2}
\end{bmatrix}
=\begin{bmatrix}
\frac{2}{s+1} & \frac{2 s-3}{(s+1)(s+2)} \\
\frac{-3}{s+1} & \frac{-2}{s+2}
\end{bmatrix}+\begin{bmatrix}
0 & 0 \\
1 & 1
\end{bmatrix} \\
=\frac{1}{s^{2}+3 s+2}\begin{bmatrix}
s\begin{bmatrix}
2 & 2 \\
-3 & -2
\end{bmatrix}+\begin{bmatrix}
4 & -3 \\
-6 & -2\end{bmatrix}
\end{bmatrix}+\begin{bmatrix}
0 & 0 \\
1 & 1
\end{bmatrix}
\end{aligned}
$$
so the realization is
$$
\begin{array}{l}
\dot{\pmb{x}}=\left[\begin{array}{cccc}
-3 & 0 & -2 & 0 \\
0 & -3 & 0 & -2 \\
1 & 0 & 0 & 0 \\
0 & 1 & 0 & 0
\end{array}\right] \pmb{x}+\left[\begin{array}{cc}
1 & 0 \\
0 & 1 \\
0 & 0 \\
0 & 0
\end{array}\right] \pmb{u}\\
~\\
\pmb{y}=\left[\begin{array}{cccc}
2 & 2 & 4 & -3 \\
-3 & -2 & -6 & -2
\end{array}\right] \pmb{x}+\left[\begin{array}{cc}
0 & 0 \\
1 & 1
\end{array}\right] \pmb{u}
\end{array}
$$
\section*{4.12}
$$
\begin{aligned}
&\hat{G}_{1}(s)=\frac{1}{s+1}\left[\begin{array}{c}
2 \\
s-2
\end{array}\right]=\frac{1}{s+1}\left[\begin{array}{c}
2 \\
-3
\end{array}\right]+\left[\begin{array}{l}
0 \\
1
\end{array}\right]\\
&\dot{\pmb{x}}_{1}=-\pmb{x}_{1}+u_{1}\\
&\pmb{y}_{c_{1}}=\left[\begin{array}{c}
2 \\
-3
\end{array}\right]\pmb{x}_{1}+\left[\begin{array}{l}
0 \\
1
\end{array}\right] u_{1}\\
&\hat{G}_{2}(s)=\frac{1}{(s+1)(s+2)}\left[\begin{array}{c}
2 s-3 \\
-2 s-2
\end{array}\right]+\left[\begin{array}{l}
0 \\
1
\end{array}\right]=\frac{1}{(s+1)(s+2)}\left[s\left[\begin{array}{c}
2 \\
-2
\end{array}\right]+\left[\begin{array}{c}
-3 \\
-2
\end{array}\right]\right]+\left[\begin{array}{l}
0 \\
1
\end{array}\right]\\
&\dot{\pmb{x}}_{2}=\left[\begin{array}{cc}
-3 & -2 \\
1 & 0
\end{array}\right] \pmb{x}_{2}+\left[\begin{array}{l}
1 \\
0
\end{array}\right] u_{2}\\
&\pmb{y}_{c_{2}}=\left[\begin{array}{cc}
2 & -3 \\
-2 & -2
\end{array}\right] \pmb{x}_{2}+\left[\begin{array}{l}
0 \\
1
\end{array}\right] u_{2}
\end{aligned}
$$
Then we can get
$$
\begin{array}{l}
\dot{\pmb{x}}=\left[\begin{array}{ccc}
-1 & 0 & 0 \\
0 & -3 & -2 \\
0 & 1 & 0
\end{array}\right] \pmb{x}+\left[\begin{array}{cc}
1 & 0 \\
0 & 1 \\
0 & 0
\end{array}\right] \pmb{u} \\
~\\
\pmb{y}=\left[\begin{array}{ccc}
2 & 2 & -3 \\
-3 & -2 & -2
\end{array}\right] \pmb{x}+\left[\begin{array}{cc}
0 & 0 \\
1 & 1
\end{array}\right] \pmb{u}
\end{array}
$$
The dimension of this realization is 3, and dimension in Problem 4.11 is 4.
\section*{4.13}
$$
\begin{aligned}
\hat{G}_{1}(s)&=\frac{1}{(s+1)(s+2)}[2 s+4 \quad 2 s-3]=\frac{1}{s^{2}+3 s+2}\left[s\left[\begin{array}{cc}
2 & 2
\end{array}\right]+\left[\begin{array}{cc}
4 & -3
\end{array}\right]\right] \\
\dot{\pmb{x}}_{1}&=\left[\begin{array}{cc}
-3 & 1 \\
-2 & 0
\end{array}\right] \pmb{x}_{1}+\left[\begin{array}{cc}
2 & 2 \\
4 & -3
\end{array}\right] \pmb{u}_{1}\\
y_{c_{1} }&=\left[\begin{array}{ll}1 & 0\end{array}\right] \pmb{x}_{1}+\left[\begin{array}{ll}0 & 0\end{array}\right] \pmb{u}_{1}\\
\hat{G}_{2}(s)&=\frac{1}{(s+1)(s+2)}[-3(s+2)-2(s+1)]+\left[\begin{array}{ll}
1 & 1
\end{array}\right]\\
&=\frac{1}{(s+1)(s+2)}\left[s\left[\begin{array}{ll}
-3 & -2
\end{array}\right]+\left[\begin{array}{ll}
-6 & -2
\end{array}\right]\right]+\left[\begin{array}{ll}
1 & 1
\end{array}\right]\\
\dot{\pmb{x}}_{2}&=\left[\begin{array}{cc}
-3 & 1 \\
-2 & 0
\end{array}\right] \pmb{x}_{2}+\left[\begin{array}{cc}
-3 & -2 \\
-6 & -2
\end{array}\right] \pmb{u}_{2}\\
y_{c_{2}}&=\left[\begin{array}{ll}1 & 0\end{array}\right] \pmb{x}_{2}+\left[\begin{array}{ll}1 & 1\end{array}\right] \pmb{u}_{2}
\end{aligned}
$$
Then we can get
$$
\begin{array}{l}
\dot{\pmb{x}}=\left[\begin{array}{cccc}
-3 & 1 & 0 & 0 \\
-2 & 0 & 0 & 0 \\
0 & 0 & -3 & 1 \\
0 & 0 & -2 & 0
\end{array}\right] \pmb{x}+\left[\begin{array}{cc}
2 & 2 \\
4 & -3 \\
-3 & -2 \\
-6 & -2
\end{array}\right] \pmb{u} \\
~\\
\pmb{y}=\left[\begin{array}{llll}
1 & 0 & 0 & 0 \\
0 & 0 & 1 & 0
\end{array}\right] \pmb{x}+\left[\begin{array}{ll}
0 & 0 \\
1 & 1
\end{array}\right] \pmb{u}
\end{array}
$$
The dimension of this realization is 4, which equal to dimension in Problem 4.11 and one more then dimension in Problem 4.12.
\section*{4.14}
$$
\begin{array}{l}
\hat{G}(s)=\begin{bmatrix}\frac{-(12 s+6)}{3 s+34}& \frac{22 s+23}{3 s+34}\end{bmatrix}=\left[\begin{array}{ll}
-4 & \frac{22}{3}
\end{array}\right]+\frac{1}{s+\frac{34}{3}}\left[\begin{array}{ll}
\frac{130}{3} & -\frac{679}{9}
\end{array}\right] \\
~\\
\begin{array}{l}
\dot{x}=-\frac{34}{3} x+\begin{bmatrix}\frac{130}{3} &-\frac{679}{9}\end{bmatrix} U \\
~\\
y=x+\begin{bmatrix}-4 & \frac{22}{3}\end{bmatrix} U
\end{array}
\end{array}
$$
\section*{4.15}
$$\hat{g}(s)=c(sI-A)^{-1}b=\frac{cR_{0}bs^{n-1}+cR_{1}bs^{n-2}+...+cR_{n-2}bs+R_{n-1}}
{\Delta s}$$
The numerator of $\hat{g}(s)$ has degree m if and only if
$cR_{n-m-1}b\neq 0$ and $ cR_{i}b=0 \quad for \quad i=1,2,...,n-m-2$.
$$
\begin{aligned}
cR_{0}b&=cb=0\\
cR_{1}b&=cAb+\alpha_{1}cb=0 \quad \Rightarrow cAb=0\\
cR_{2}b&=cA^{2}b+\alpha_{2}cR_{1}b=0 \quad \Rightarrow cA^{2}b=0\\
&\hspace{2cm}\vdots\\
cR_{n-m-2}b&=cA^{n-m-2}b+cR_{n-m-3}b=0 \quad \Rightarrow cA^{n-m-2}b=0\\
cR_{n-m-1}b&=cA^{n-m-1}b+cR_{n-m-2}b\neq 0 \quad \Rightarrow cA^{n-m-1}b\neq 0
\end{aligned}
$$
Then $\hat{g}(s)$ has m zeros if and only if $cR_{n-m-1}b\neq 0$ and $ cR_{i}b=0\quad for \quad i=1,2,...,n-m-2$.
\section*{4.16}
(1)
$$
\begin{aligned}
&\begin{array}{l}
\qquad \begin{array}{l}
\dot{x}_{1}=x_{2} \Rightarrow x_{1}(t)=\int_{0}^{t} x_{2}(t) d t+x_{1}(0) \\
\dot{x}_{2}=t x_{2} \Rightarrow x_{2}(t)=x_{2}(0) e^{0.5 t^{2}}
\end{array} \\
~\\
\text { We have } \\
X(0)=\left[\begin{array}{l}
1 \\
0
\end{array}\right] \Rightarrow X(t)=\left[\begin{array}{l}
1 \\
0
\end{array}\right] \\
 X(0)=\left[\begin{array}{l}
0 \\
1
\end{array}\right] \Rightarrow X(t)=\left[\begin{array}{c}
\int_{0}^{t} e^{0.5 \tau^{2} d \tau} \\
e^{0.5 t^{2}}
\end{array}\right]
\end{array}\\
&\text { Then the fundamental matrix and transition matrix are}\\
&\begin{array}{l}
X(t)=\left[\begin{array}{ll}
1 & \int_{0}^{t} e^{0.5 \tau^{2} d \tau} \\
0 & e^{0.5 t^{2}}
\end{array}\right] \\
\Phi\left(t, t_{0}\right)
=\left[\begin{array}{ll}
1 & \int_{0}^{t} e^{0.5 \tau^{2} d \tau} \\
0 & e^{0.5 t^{2}}
\end{array}\right]\left[\begin{array}{ll}
1 & \int_{0}^{t_{0}} e^{0.5 \tau^{2} d \tau} \\
0 & e^{0.5 t_{0}^{2}}
\end{array}\right]^{-1}\\
=\left[\begin{array}{ll}
1 & \int_{0}^{t} e^{0.5 \tau^{2} d \tau} \\
0 & e^{0.5 t^{2}}
\end{array}\right]\left[\begin{array}{ll}
1 & -e^{-0.5t_{0}^{2}}\int_{0}^{t_{0}} e^{0.5 \tau^{2} d \tau} \\
0 & e^{-0.5 t_{0}^{2}}
\end{array}\right]
=\left[\begin{array}{ll}
1 & e^{-0.5 t_{0}^{2} d \tau} \int_{t_{0}}^{t} e^{0.5 \tau^{2} d \tau} \\
0 & e^{0.5\left(t^{2}-t_{0}^{2}\right)}
\end{array}\right]
\end{array}
\end{aligned}
$$
(2)
$$
\begin{aligned}
&\begin{array}{l}
\dot{x}_{1}=-x_{1}+e^{2 t} x_{2}(t)=-x_{1}+e^{t} x_{2}(0) \Rightarrow
x_{1}(t)=0.5 x_{2}(0)\left(e^{t}-e^{-t}\right)+x_{1}(0) e^{-t} \\
\dot{x}_{2}=-x_{2} \Rightarrow x_{2}(t)=x_{2}(0) e^{-t}
\end{array}\\
&\text{Then}\\
&X(0)=\left[\begin{array}{l}
1 \\
0
\end{array}\right] \Rightarrow X(t)=\left[\begin{array}{c}
e^{-t} \\
0
\end{array}\right]\\
& X(0)=\left[\begin{array}{l}
0 \\
1
\end{array}\right] \Rightarrow X(t)=\left[\begin{array}{c}
0.5\left(e^{t}-e^{-t}\right) \\
e^{-t}
\end{array}\right]\\
&\text { Then the fundamental matrix and transition matrix are}\\
&\begin{array}{l}
X(t)=\left[\begin{array}{cc}
e^{-t} & 0.5\left(e^{t}-e^{-t}\right) \\
0 & e^{-t}
\end{array}\right] \\
\Phi\left(t, t_{0}\right)=\left[\begin{array}{cc}
e^{-t} & 0.5\left(e^{t}-e^{-t}\right) \\
0 & e^{-t}
\end{array}\right]\left[\begin{array}{cc}
e^{-t_{0}} & 0.5\left(e^{t_{0}}-e^{-t_{0}}\right) \\
0 & e^{-t_{0}}
\end{array}\right]^{-1} \\
=\left[\begin{array}{cc}
e^{t_{0}-t} & 0.5 e^{-t}\left(e^{t_{0}}-e^{-3 t_{0}}\right)+0.5\left(e^{t}-e^{-t}\right) e^{t_{0}} \\
0 & e^{t_{0}-t}
\end{array}\right]
\end{array}
\end{aligned}
$$
\section*{4.17}
$$
\frac{\partial }{\partial t}X^{-1}(t)$$
$$
\frac{\mathrm{d} }{\mathrm{d} t}(X(t)X^{-1}(t))=\dot {X}(t)X^{-1}(t)+X(t)\frac{\mathrm{d} }{\mathrm{d} t}X^{-1}(t)=0\\$$
$$\Downarrow\\$$
$$
\frac{\mathrm{d} }{\mathrm{d} t}X^{-1}(t)=-X^{-1}(t)\dot {X}(t)X^{-1}(t)=-X^{-1}(t)A(t) X(t)X^{-1}(t)=-X^{-1}(t)A(t)$$
Then we have
$$\frac{\partial }{\partial t}\Phi(t_{0},t)=-X(t_{0})X^{-1}(t)A(t)=-\Phi(t_{0},t)A(t)$$
\section*{4.18}
$$
\frac{\partial }{\partial t}\Phi(t,t_{0})=
\begin{bmatrix}
\frac{\partial }{\partial t}\phi_{11}&\frac{\partial }{\partial t}\phi_{12}\\
\frac{\partial }{\partial t}\phi_{21}&\frac{\partial }{\partial t}\phi_{22}
\end{bmatrix}=
\begin{bmatrix}
a_{11}&a_{12}\\
a_{21}&a_{22}
\end{bmatrix}
\begin{bmatrix}
\phi_{11}&\phi_{12}\\
\phi_{21}&\phi_{22}
\end{bmatrix}
$$
Then
$$
\begin{aligned}
\frac{\partial}{\partial t}(\operatorname{det} \Phi)&=\frac{\partial}{\partial t}\left(\phi_{11} \phi_{22}-\phi_{21} \phi_{12}\right)\\
&=\frac{\partial}{\partial t}\phi_{11} \phi_{22}-\frac{\partial}{\partial t}\phi_{21} \phi_{12}+\phi_{11} \frac{\partial}{\partial t}\phi_{22}-\phi_{21} \frac{\partial}{\partial t}\phi_{12}\\
&=\left(a_{11} \phi_{11}+a_{12} \phi_{21}\right) \phi_{22}-\left(a_{11} \phi_{12}+a_{12} \phi_{22}\right) \phi_{21}+\phi_{11}\left(a_{21} \phi_{12}+a_{22} \phi_{22}\right)-\phi_{12}\left(a_{21} \phi_{11}+a_{22} \phi_{21}\right)\\
&=\left(a_{11}+a_{22}\right)\left(\phi_{11} \phi_{22}-\phi_{21} \phi_{12}\right)\\
&=\left(a_{11}+a_{22}\right) \operatorname{det} \Phi
\end{aligned}
$$
Then
$\operatorname{det} \Phi\left(t, t_{0}\right)=c e^{\int_{t_{0}}^{t}\left(a_{11}(\tau)+a_{22}(\tau)\right) d \tau}$ and $ \Phi\left(t_{0}, t_{0}\right)=I$, so we can get $c=1 ,\operatorname{det} \Phi\left(t, t_{0}\right)= e^{\int_{t_{0}}^{t}\left(a_{11}(\tau)+a_{22}(\tau)\right) d \tau}$
\section*{4.19}
$$
\Phi(t_{0},t_{0})=\begin{bmatrix}
\phi_{11}(t_{0},t_{0})&\phi_{12}(t_{0},t_{0})\\
\phi_{21}(t_{0},t_{0})&\phi_{22}(t_{0},t_{0})
\end{bmatrix}=I
$$
Then $\phi_{21}(t_{0},t_{0})=0,\phi_{22}(t_{0},t_{0})=I$.
$$
\frac{\partial }{\partial t}\Phi(t,t_{0})=
\begin{bmatrix}
\frac{\partial }{\partial t}\phi_{11}(t,t_{0})&\frac{\partial }{\partial t}\phi_{12}(t,t_{0})\\
\frac{\partial }{\partial t}\phi_{21}(t,t_{0})&\frac{\partial }{\partial t}\phi_{22}(t,t_{0})
\end{bmatrix}=
\begin{bmatrix}
A_{11}(t)&A_{12}(t)\\
0&A_{22}(t)
\end{bmatrix}
\begin{bmatrix}
\phi_{11}(t,t_{0})&\phi_{12}(t,t_{0})\\
\phi_{21}(t,t_{0})&\phi_{22}(t,t_{0})
\end{bmatrix}
$$
Then
$$
\begin{aligned}
\frac{\partial }{\partial t}\phi_{11}(t,t_{0})&=A_{11}(t)\phi_{11}(t,t_{0})+A_{12}(t)\phi_{21}(t,t_{0})\\
\frac{\partial }{\partial t}\phi_{21}(t,t_{0})&=A_{22}(t)\phi_{21}(t,t_{0})\\
\frac{\partial }{\partial t}\phi_{22}(t,t_{0})&=A_{22}(t)\phi_{22}(t,t_{0})
\end{aligned}
$$
The equation $\frac{\partial }{\partial t}\phi_{22}(t,t_{0})=A_{22}(t)\phi_{22}(t,t_{0})$ with $\phi_{22}(t_{0},t_{0})=I$ has the unique solution $\phi_{22}(t,t_{0})$.
The equation $\frac{\partial }{\partial t}\phi_{21}(t,t_{0})=A_{22}(t)\phi_{21}(t,t_{0})$ with $\phi_{21}(t_{0},t_{0})=0$ has the unique solution $\phi_{21}(t,t_{0})\equiv 0$.
So $\frac{\partial }{\partial t}\phi_{11}(t,t_{0})=A_{11}(t)\phi_{11}(t,t_{0})+
A_{12}(t)\phi_{21}(t,t_{0})=A_{11}(t)\phi_{11}(t,t_{0})$
\section*{4.20}
$$
\begin{array}{l}
\dot{x}_{1}=-\sin t x_{1} \Rightarrow x_{1}(t)=x_{1}(0) e^{\cos t} \\
\dot{x}_{2}=-\cos t x_{2} \Rightarrow x_{2}(t)=x_{2}(0) e^{-\sin t}
\end{array}
$$
then we have
$$
\begin{aligned}
X(0)&=\left[\begin{array}{l}1 \\ 0\end{array}\right] \Rightarrow X(t)=\left[\begin{array}{c}e^{\cos t} \\ 0\end{array}\right] \\
X(0)&=\left[\begin{array}{l}0 \\ 1\end{array}\right] \Rightarrow X(t)=\left[\begin{array}{c}0 \\ e^{-\sin t}\end{array}\right]
\end{aligned}
$$
Then the fundamental matrix is
\[X(t)=\left[\begin{array}{cc}e^{\cos t} & 0 \\ 0 & e^{-\sin t}\end{array}\right] \]
So the state transition matrix is
\[\Phi\left(t, t_{0}\right)=\left[\begin{array}{cc}e^{\cos t} & 0 \\ 0 & e^{-\sin t}\end{array}\right]\left[\begin{array}{cc}e^{-\cos t_{0}} & 0 \\ 0 & e^{\sin t_{0}}\end{array}\right]=\left[\begin{array}{cc}e^{\cos t-\cos t_{0}} & 0 \\ 0 & e^{-\sin t+\sin t_{0}}\end{array}\right]\]
\section*{4.21}
Because
$\dot{X}(t)=Ae^{At}Ce^{Bt}+e^{At}Ce^{Bt}B=AX+XB$
and
$X(0)=e^{A\dot0}Ce^{B\dot0}=C$
,so $X(t)=e^{At}Ce^{Bt}$ is the solution.
\section*{4.22}
We can get the solution of $\dot{A}(t)=A_{1}A(t)+A(t)(-A_{1})$ is $A(t)=e^{A_{1}t}A(0)e^{-A_{1}t}$ from the conclusion of Problem 4.20.\\
$$
\begin{aligned}
det(\lambda I-A(t))&=det(e^{A_{1}t}\lambda Ie^{-A_{1}t}-A(t))\\
&=det(e^{A_{1}t}(\lambda I-A(0))e^{-A_{1}t})\\
&=det e^{A_{1}t} det e^{-A_{1}t} det(\lambda I-A(0))\\
&=det(\lambda I-A(0))
\end{aligned}
$$
So the eigenvalues of $A(t)$ are incdependent of $t$.
\section*{4.23}
Define $T=2\pi$, then
\[X(t)=\left[\begin{array}{cc}e^{\cos t} & 0 \\ 0 & e^{-\sin t}\end{array}\right],
X(t+T)=\left[\begin{array}{cc}e^{\cos (t+2\pi)} & 0 \\ 0 & e^{-\sin (t+2\pi)}=X(t)\end{array}\right]
\]
$$X(t+T)=X(t)e^{\bar{A}t}\Rightarrow \bar{A}=0$$
Then
\[P(t)=e^{\bar{A}t}X^{-1}(t)=\left[\begin{array}{cc}e^{-\cos t} & 0 \\ 0 & e^{\sin t}\end{array}\right]\]
Let $\bar{\pmb{x}}=P(t)\pmb{x}$, then the state equation is
$\dot{\bar{\pmb{x}}}=\bar{A}\pmb{x}=0$.
\section*{4.24}
$$X(t)=e^{At}\quad \Rightarrow X^{-1}(t)=e^{-At}$$
Let $P(t)=e^{\bar{A}t}X^{-1}(t)=X^{-1}(t)=e^{-At}$ and $\bar{\pmb{x}}=P(t)\pmb{x}$, then we can get
$$\bar{A}(t)=0 \quad \quad \bar{B}(t)=e^{-At}B \quad \quad \bar{C}(t)=Ce^{At}$$
\section*{4.25}
time-varying realization:
$$g(t-\tau)=(t-\tau)^{2}e^{\lambda(t-\tau)}=
\begin{bmatrix}
t^{2}e^{\lambda t}&-2te^{\lambda t}&e^{\lambda t}
\end{bmatrix}\begin{bmatrix}
e^{-\lambda \tau}\\ \tau e^{-\lambda \tau} \\ \tau^{2}e^{-\lambda \tau}
\end{bmatrix}$$
Thus a time-varying realization is
$$
\begin{aligned}
\dot{\pmb{x}}&=\begin{bmatrix}
0&0&0\\0&0&0\\0&0&0
\end{bmatrix}\pmb{x}+
\begin{bmatrix}
e^{-\lambda t}\\ t e^{-\lambda t} \\ t^{2}e^{-\lambda t}
\end{bmatrix}u\\
y(t)&=\begin{bmatrix}
t^{2}e^{\lambda t}& -2t e^{\lambda t} &e^{\lambda t}
\end{bmatrix}\pmb{x}
\end{aligned}
$$
time-invariant realization:
The Laplace transform of the impluse response is
$$\hat{g}(s)=\frac{2}{(s-\lambda)^{2}}=\frac{2}{s^{3}-3\lambda s^{2}+3\lambda^{2}s+\lambda^{3}}$$
Then we can get
$$
\begin{aligned}
\dot{\pmb{x}}&=\begin{bmatrix}
3\lambda & -3\lambda^{2}&\lambda^{3}\\1&0&0\\0&1&0
\end{bmatrix}\pmb{x}+
\begin{bmatrix}
1\\ 0 \\0
\end{bmatrix}u\\
y(t)&=\begin{bmatrix}
0& 0 &2
\end{bmatrix}\pmb{x}
\end{aligned}
$$
\section*{4.26}
$$g(t,\tau)=sint e^{-t} e^{\tau}cos \tau$$
Thus a time-varying realization is
$$
\begin{aligned}
\dot{x}&=0 x+e^{t}cost u\\
y&=sint e^{-t}u
\end{aligned}
$$
Because $g(t,\tau)$ can not be expended as $g(t-\tau)$ , it is not possible to find a time-invariant state equation realization.
\end{document}